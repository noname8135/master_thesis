\chapter{Flow entry verification}
This work focuses on the problem that a compromised switch will bring and presents the method to detect such a switch. Our method emphasizes on scanning though the whole network with less packets and detection efficiency. In this chapter, we will make assumptions to create a scenario and how the problems can be solved using the feature of SDN. We will list the threat models and explain them along with their attack scenarios and detection algorithms.

\section{Threat models and Attack scenario}
In this work, we assume the following scenario of compromising a switch in the threat model:
\begin{enumerate}
\item
Only one OpenFlow switch is compromised. No cooperation among multiple compromised switches for attacks will happen. 
\item
The switches are able to access the Internet. 
\item
Other parts of the network such as the controller, other switches and hosts function normally. Any potential flaw is unintentional and out of the scope of this work.
\item
An attacker cannot totally change the way of switch processing or core mechanism, but only perform the attack by modifying flow entries.
\item
Initially, the network is all clean, and nothing is compromised. The attacks take place some time after the whole network is established.
\end{enumerate}

Other attack scenarios beyond the assumptions such as multiple compromised switches, as well as possible circumvention, will be discussed in Section~\ref{Further_discussion}.

\section{Rapid detection method of disobedient forwarding}
Our method aim to detect if the flow entries of a switch work as expected. It is inspired by \cite{CKGL15}. The method in this work has two main enhancements. First, it reduces the number of detection packets required, and therefore increases the efficiency significantly. Second, no existing flow entries is modified. Only some temporary entries will be added and will timeout after the detection process is over, which has little influence to the whole network and is easy to clean up. 

\subsection{Terminology}

\begin{description}%[font={\small}]

\item
[Aggregation conditions]:
The conditions for an entry to be in the same aggregated group. It is defined to ensure a valid detection packet can be forged and sent through the switches in the group successfully. There are several of them:
\begin{enumerate}[label={\arabic*)}]
\item
In the same group, the flow entries on different switches either have exactly the same match fields and values, or have no common match fields.
\item
The route for a detection packet of an aggregation group must not contain any loop.
\item
For a flow entry \textit{A} on an arbitrary switch, if there exist another flow entry \textit{B} on another switch such that \textit{A} and \textit{B} have the same match field and value, then \textit{A} might belong to more than one aggregated group. Otherwise entries belong to one aggregated group.

\end{enumerate}


\item
[Aggregated groups]: 
A set of entries that satisfied the aggregation condition. An entry in the group has forwarding action either to the switch that the other entry is on or a host. Each group has an integer number as an identifier.

\item 
[Aggregation tree]
A tree format data structure for packet traversal route of an aggregated group. It treats the switches that the entries of the aggregated group are on as vertex, and the forwarding action of those entries as edges. It contains:
\begin{enumerate}
\item
[starting switch]:
A switch as a start point for traversal.
\item
[splitting switches]:
The switches who has more than one children. 
\item
[leaf switches]:
The switches who are the leaves of the aggregation tree.
\end{enumerate}
\item 

\item
[Detection packets]:
It is forged according to the match fields of the flow entries in an aggregated group. The vid field of each detection packet is set to the group id it associates to.
\end{description}

Further explanation of these setups will be explained along in next subsection.

\subsection{The detection method}
\label{Detection_method}

The main idea of the method is to assemble a packet that will go through a sequence of switches by matching the match fields in the flow entries of these switches. Then the packet will be sent into the network, go through the switches, and should be sent back to the controller from expected switches finally if nothing goes wrong. Therefore, the detection packet can check whether the matched flow entries on these switches work as expected or not. For this purpose, we need to find the path that a detection packet should traverse, find the flow entry on each switch with which the packet will be matched, and set the fields in the packet so that it will pass through the switches in order. Because the controller has the network-wise visibility and the policy of all the flow entries, it is able to decide the switches to be involved in an aggregation tree and the detection packet in each run of detection. 

\begin{figure}[H]
\begin{center} 
\includegraphics[width=1\textwidth]{figures/flow_entry_detection_flowchart.pdf}
\end{center}
\caption{The flow chart of the flow entry detection process.}
\label{flow_entry_detection_flowchart}
\end{figure}

The first aggregation condition is the essential idea of the method: aggregating flow entries this way allows us to forge a packet with multiple fields that match the entries inside the same aggregated group. The second condition is to eliminate the cycle in the forwarding path of a detection packet. Since a switch has at most one flow entry in an aggregated group, a detection packet will not go through a switch more than once. Otherwise it might causes loop. The third aggregation condition is also for eliminating loops, we will talk more about it in \ref{Aggregated_group_finding}.

A detection packet is needed for each aggregated group. In other words, minimizing the number of aggregated groups means minimizing the number of detection packets needed, resulting in fast detection. In order to achieve this, we try to increase the number of entries that one detection packet goes through. On a switch, there might be a collection of entries $E$ that suit the aggregation conditions. We can add a new flow entry with multiple forwarding actions. It will duplicate the detection packet and forward them according to the action set. Therefore, the switches the entries in an aggregated group are on and their forwarding actions will form a tree. More detail will be stated in \ref{Aggregated_group_finding}.

As the abstraction of the problem, we treat switches as vertex and forwarding actions of entries as edges. Our goal is to find the minimum number sets of aggregated group such that all entries belong to at least an aggregated group. It forms a complex DAG problem and is more complex than the longest path problem \cite{DMR97,RU04}, which is NP-complete and hard to find a approximation algorithm. Starting from an arbitrary switch, we use DFS to traversal and compose aggregated groups one by one until all the entries belong to at least one group. It can be done in a reasonable amount of time, which will be listed in \ref{Implementation_and_Evaluation}, under the scale of experimental environments.


\subsection{Aggregate group finding}
\label{Aggregated_group_finding}

The flow chart of the flow entry detection process in the controller is shown in Figure~\ref{flow_entry_detection_flowchart}. Let $S=\{s_1,s_2,\ldots,s_n\}$ be the set of switches under the control of a controller, and $f(s_i)$, where $i=1\ldots n$, represents the flow entries on $S_i$. Let $F=\cup_{i=1}^n f(s_i)$, i.e., the set of all the flow entries. In the first step of the flow chart, we attempt to find $A$, overlapping disjoint sets of $F$, such that each element $a$ (i.e. aggregation group) belong to $A$, all entries $E$ in $a$ satisfy the aggregation conditions, noted . 

When finding an aggregated group, we start from an arbitrary switch as the starting switch. While we search for $E$ on a switch $S_x$, if we find a such $E$, we follow its forwarding action and move on to the destination. 



otherwise if we cannot find such $E$, we return to last depth, which the parent node of this switch in aggregation tree. If we find an entry 

unless the detection packet has no more field available. 



switch and try to find an entry which also meets aggregation condition and also has an output action action to the third switch, and so on. Eventually when we find a switch that does not contain any flow entry that fits the requirement, and the detection process in this aggregated group ends. Because all flow entries will be aggregated to some aggregated group, all the flow entries will be verified after all the aggregated groups have been explored.
---------------




After finding an aggregated group, the controller will send a detection packet with PACKET\_OUT to the starting switch, and the detection packet will pass the through the switches in the aggregation trees. The detection packets should also meet the dependency requirement stated in the third paragraph of \ref{SDN and OpenFlow}; otherwise, it will not be matched by any flow entry. When it arrives at a splitting switch, it will be duplicated and sent to more than one switch due to the additional flow entry we install on splitting switch. Eventually, when the detection packet reach the last entry in the aggregated group, it will be sent to an end point, which is either a switch that contain no flow entry that matches the aggregation conditions or a host depending on the output port of the last entry. If the end point is a switch, the packet will fail to match any flow entry on that switch and be sent back to the controller due to the table-miss entry, and the switch will become a leaf switch of the aggregation tree. If it is a host, the controller installs a new flow entry on the leaf switch the last entry is on. Either case will notifying the controller that the detection process of this aggregated group is carried out as expected.

Figure~\ref{aggregated_group} illustrates a realistic example with an aggregated group and the content of a detection packet. The flow entries connected by arrows belong to the same group. we have a controller, three switches, $s_a$,$s_b$ and $s_c$, and a detection packet.

XXXXXXX


After, the controller sends the detection packet to starting switch. In the example figure,  Only the entries that will be matched are shown. In $s_a$, there is a flow entry that contains match field \texttt{eth\_dst}=\texttt{a2:b4:c6:d8:e0:f1} and an output action to $s_b$. In $s_b$, the flow entry has match fields \texttt{eth\_type}=\texttt{0x800}, \texttt{ipv4\_src}=\texttt{140.123.103.123} and an output action to $s_c$. The \texttt{eth\_type}=\texttt{0x800} is required for a valid IP match field. According to the match fields of these flow entries in the aggregated group, the forged detection packet will contain \texttt{eth\_dst} as \texttt{a2:b4:c6:d8:e0:f1}, \texttt{ipv4\_src} as \texttt{140.123.103.123}.  

\begin{figure}[H]
\begin{center}
\includegraphics[width=1\textwidth]{figures/aggregated_group.png}
\end{center}
\caption{An aggregated group and constructed packet. \red{wrong figure}}
\label{aggregated_group}
\end{figure}

The process of generating a packet is done along with defining one aggregated group. Its pseudo-code is as follows:

\begin {tcolorbox}[blanker,float=tbp,
grow to left by=1cm, grow to right by=1cm]
\begin{algorithm}[H]

  \caption{Packet generating process.}
  \begin{algorithmic}[1]
    \Require
      Set of switch, $switches$;  \newline
      Each switch is a set of entries, $switch$;  \newline
      Info of an entry, including match field, value and destination switch, $entry$;  \newline
      Visited status of every switch in one group finding process, cleared every round, $visited\_switch$;  \newline
      An under-constructing packet that is initially empty and send after one round of group finding is done, $packet$; \newline
      In-Degree of every switch that will be alternated dynamically, $switch_in_degree$
      
    \Function{find\_aggregated\_groups}{$switches$}
      \State $\textit{group\_id} \gets 0$;
      \While{not all $switch$ is empty}  
            \State $\textit{starting\_switch} \gets A\;switch\;with\;max\;\Call{heuristic}{\textit{switch}}\;\forall\;\textit{switch} \in \textit{switches}$;
            \State $\textit{packet} \gets empty\;dictionary$;   // Cleared per group finding round
            \State $packet[vid] \gets \textit{group\_id}$;   // Use vid as group identifier 
            \State $group\_id \gets \textit{group\_id} + 1$;
            \State $\textit{packet} \gets \Call{find\_one\_group}{\textit{starting\_switch}, \textit{packet},$visited\_switch$}$;
            \State $Send\;\textit{packet}\;with\;PACKET\_OUT$;
      \EndWhile
    \EndFunction
    \State
    \Function{heuristic}{$switch$}
      \State \Return $value$;
    \EndFunction
    \State
  \algstore{find_group}
  \end{algorithmic}
\end{algorithm}
\end{tcolorbox}

\begin {tcolorbox}[blanker,float=tbp,
grow to left by=1cm, grow to right by=1cm]
\begin{algorithm}[H]
  \begin{algorithmic}[1]
  \algrestore{find_group}
    \Function{find\_one\_group}{switch$, packet$, $visited\_switch$}
      \State $\textit{best\_entry} \gets \textit{entry}\;with\;max\; \Call{heuristic}{\textit{entry}[\textit{dst\_switch}]}\;\forall\;\textit{entry}\;in\;\textit{switch}$
      \State $such\;that\;\textit{entry}[\textit{dst\_switch}] \notin \textit{visited\_switch}\;and\;\textit{entry}\;fit\;in\;group$;  //get best entry that fits the aggregation conditions;
      \If{$best\_entry$}  //found an entry that satisfy aggregation condition
        \State $\textit{packet}[\textit{match\_field}] \gets \textit{match\_value}$;  //add match field to detection packet
        \State Add $best\_entry$[$dst\_switch$] to $visited\_switch$;
        \State \Return \Call{packet\_gen}{$match\_field$, $match\_value$, $action$, $packet$};  //no entry in the switch satisfies aggregation condition of this group
      \EndIf
      \State \Return $packet$;
    \EndFunction
  \end{algorithmic}
\end{algorithm}
\end{tcolorbox}


\subsection{Further discussion of attack scenario and our method}
\label{Further_discussion}
On splitting switches and leaf switches, we install additional flow entries. However, this will make the detection packet matches the newly installed flow entries rather than the original ones. XXXXXXXXXXXXXX
With egress processing enabled, we are able to add the new flow entry in egress table, 


As the result, the original entries was matched, and no entry is missed. 


XXXXXXXX\cite{OF_SPEC_15}

The trait of match field mentioned in third paragraph of \ref{SDN and OpenFlow} makes match field aggregation much more complicated. These kind of situations can still be covered by our method with some modification if we manage to resolve the complex conjunctive conditions. Take the entry with multiple match fields for example, if all the fields and values match the under-constructing packet, we process it according to first aggregation rule. Otherwise we fill the packet with all of those match fields. If any match field is already taken with different value, it will be ignored for this turn of aggregation process and fill the match fields in another packet that has no conflict field. However, the main purpose of our method is to show the effectiveness of flow entry aggregation method. We will demonstrate with the most simple condition, which is that every entry has only one matching field and use only one flow table.

It is certainly possible that more than one switch are made by the same manufacturer or have the same software version in the same network, so multiple switches share the same vulnerabilities and may be compromised simultaneously. However, cooperation between multiple compromised switches complicates the scenario significantly. In order to simplify the case as a starting point for developing the detection method, we assume only one switch inside the network is compromised.

Suppose an attacker is able to add, remove, or modify the entries in the flow tables of a compromised
switch without notifying the controller, so packets may be forwarded to an undesired destination. The proposed method is intended to detect this behavior with high efficiency. In this method, only output action of packets is considered. Other actions such as dropping packets, setting field, changing TTL is not included \sout{\red{Cann't they (e.g., dropping packets) be detected?}}. Although there are might be reasonable ways to deal with these actions, for example, to detect packet dropping, we can add timeout checking function, such detection function is irrelevant to our main method and is not implemented in our work.

In this detection method, there is another important consideration: the controller must maintain unpolluted information of all flow entries. It is not reliable for a controller to obtain flow entry information by querying switches because a switch may be malicious and forge a fake response that gives false information about the flow entries. That is the reason for the last statement of attack model. Also mind that the detection packet should not be distinguishable from normal packet, otherwise the attacker will be able to circumvent the detection method. 
------

 