\begin{abstract}
\label{sec:abstract}
Software defined network (SDN) is a next-generation concept that allows network administrators to manage network flows a lot easier. It is programmable, centrally managed, and flexible with topology alteration. However, these new features also lead to new security problems. Applications, controllers, Openflow switches, topology managing mechanism, there are a lot of newly-introduced vectors for us to concern about. Although protection mechanisms such as TopoGuard and FortNox have been proposed, there are still more of attack posibilities and countermeasures in different scenarios left to be discovered. Furthermore, as SDN evolves, it will definitely bring more new security issues. In this paper, we analyze possibilities of attacks when a switch is compromised and create a disobedient forwarding detection method. To enhance the detection efficiency and minimize the additional network traffic, we try to reduce the number of detection packets by aggregating the flow entries in a short amount of computation time. The experimental result demonstrates that our method is effective under network topologies with balance structure, and the scale of the network topology does not affect the efficiency of our method significantly.
\end{abstract}