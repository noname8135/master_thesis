\chapter{Introduction}
\label{chap:intro}
\setcounter{page}{1}
\pagenumbering{arabic}

Nowadays, as the popularity of public clouds like Google cloud, Microsoft Azure, Amazon EC2 increase, we can see how cloud computing offer a new way of deploying applications and service. While people try to move everything onto the cloud, the growth of volume and complexity of data center network (DCN) are getting out of hand, and server virtualization is also becoming more and more common. Networking organization are under increasing pressure to be more efficient, agile, and maintainable than is possible with the traditional approach to networking.

In traditional network, most of the network functionality heavily rely on hardware because they are implemented in network devices. They are under the control of manufacturers. As a result, the evolution of those functionality will be limited. Furthermore, implementing a network-wide policy requires configuring at the device-level. And similarly, tasks such as provisioning, change management, and de-provisioning are also very time-consuming, error-prone and require a lot of manpower. It will surely be a formidable job for network administrators to manage a large scale network with traditional network. Scalability is also a problem, link oversubscription is no longer reliable enough to keep up with the growing demand on data centers \cite{TTN}.

Software defined network (SDN) is a dynamic, manageable, cost-effective, and adaptable network network structure. It gains great popularity among enterprises as well as academia recent years. One key concept of SDN is separating the control plane from data plane, it is centrally controllable by software application. Comparing to legacy network, it is more flexible, maintainable and agile. Also, components of legacy network such as a regular switch is totally compatible within a SDN network structure.

Nevertheless, new technology often comes with new security problems. In addition to switches and hosts, which of course are potential targets for hackers, SDN uses controllers to realize centralized control. If the applications in controllers are compromised, the whole network might fall into attacker's hand \cite{TSA}. We will talk more about these issues in later chapters.

Beside those hardware components, attackers might want to leverage some crucial software components of SDN, like OpenFlow protocol, applications in controller, and topology services. The topology-related services  provides a manner to adjust to the modification of network topology automatically. Nevertheless, some parts of it is still not very mature and has proven to be insecure. It may lead to Man In The Middle attack (MITM) or Deny of Service (DoS) in a similar way to how it could happen in regular network \cite{ASO}. In this thesis, we will be focusing on those topology-related issues.

\emph{The motivation of our work} is that, although lots of works has been done to deal with all these security problems in SDN, some aspect are missing. For example, in \cite{HXWG_15}, Hong et al. propose attacks that poison network visibility and its countermeasure, but did not consider the situation that switches are compromised. And to the best of our knowledge, there has not been an effective way to detect if there is any switch being compromised in the network. 

With the program-configurable and manageable traits of SDN, one is able to implement network function like IDS, firewall, WAF etc. We believe it is possible to implement similar defensive solution with the aids of those SDN properties. \emph{The main goal of this thesis} is to promotes existing method and try to propose a new detection method using the standard functionality of OpenFlow switch. With this work, we hope we are able to set an example to inspire others and draw more attention from the community to concern about the security issues in SDN, rise the security awareness of SDN users, and ultimately resulting in a more mature SDN environment.

During the research, we study the specification of components as well as the potential threats in SDN. After discussing about those attacks, we try to cover the situations that were left by the works of the others. Then some threat models and the frame of our innovative method is proposed hypothetically. 
In the experiments, we found that XXXXXXXXXXXXXXXXXXXXXXXXXXXXXXXXXXXXXXXXX. Eventually, we figure out a way to detect whether is an switch acts like the way it should be. Finally, we will assess how effective our method is.

The \emph{main contributions of this paper} are as follow:

\begin{enumerate}
\item
Analysis several types of attacks that influence the visibility of network.
\item
Discusses about previously-created counter measurements.
\item
Create an innovative detection measure and improve currently-existing method.
\item
Evaluate our method under different conditions.
\end{enumerate}

The following chapters in this thesis will be: Chapter 2 gives detail background knowledge of the used technology, discuss about possible threats and countermeasure. Chapter 3 is about our threat model and the theory of our own detection method. 
-----------------------------------------------
Chapter 4 contains the experimental details including setup, considerations, simulations of attacks and evaluation methods. In Chapter 5, the proposed method will be evaluated under different conditions. Finally, the conclusion and future expectation of this work will be in Chapter 6.