\chapter{Introduction}
\label{chap:intro}
\setcounter{page}{1}
\pagenumbering{arabic}

Nowadays, as the popularity of public clouds like Google cloud, Microsoft Azure, Amazon EC2 increase, we can see how cloud computing offers a new way of deploying applications and services. While users tend to move everything onto the clouds, handling the growth of volume and complexity of data center network (DCN) is getting challenging. Consequently, networking organizations are under increasing pressure of being more efficient, agile and maintainable. In traditional networks, implementing a network-wide policy requires configuring at the device level. The process is time-consuming, error-prone and manpower-heavy, and it is a formidable job for network administrators to manage a large network.

Software defined network (SDN) is a dynamic, programmable, cost-effective network structure that gains great popularity in the industry and academia in recent years {\color{red} Refer to some popular SDN tutorial papers here}. In SDN, a controller centrally manage a number of switches in a consistent manner, and network applications can be developed on the controller for flexible management. The controller is able to control all network flow by configuring flow rules inside the switches. Also, components of legacy network such as a regular switch is totally compatible within a SDN network structure {\color{red} Add a reference to justify this sentence}.

Nevertheless, SDN often comes with new security problems {\color{red} Refer to some popular SDN security survey papers here}. There are also some new mechanisms such as topology discovery, host management, protocols and APIs for the communication between entities. With so many new elements introduced, there will be more potential security issues that need to be taken care of in various ways. {\color{red} Briefly summarize the categories of security issues in SDN here.}

In this work, we present a method to address xxx {\color{red} The issues to be addressed in this work}. For example, in \cite{HXWG15}, Hong et al. propose attacks that poison network visibility and its countermeasure, but they did not consider the situation that switches are compromised. to the best of our knowledge, although some protection method are proposed, there has not been an effective way to detect if there is any switch being compromised in the network. With the program-configurable trait of SDN, we believe it is possible to implement defensive solution with the aids of SDN properties. We hope we are able to set an example to inspire others and draw more attention from the community to concern more about the security issues in SDN, and ultimately resulting in a more mature SDN environment.

During the research, we study the specification of components as well as the potential threats in SDN. After discussing about those attacks, we try to cover the situations that were left by others. In this work, there are \emph{two main goals}: the first one is to improve an existing switch detection method. The detection method is able to detect whether a flow entry works as expected. However, it can only detect one flow entry at a time, which does not seem to be efficient enough. To solve this problem, we try to aggregate the match fields to reduce the number of detection packets. The second goal is to propose a new detection method that is able to detect a fabricated link caused by LLDP packets manipulation. The main idea behind it is to measure the round trip time of LLDP packets. If an intermediate switch is compromised and manipulates the LLDP packet, it is very likely to increase the round trip time. Nevertheless, time measuring will be significantly influenced by traffic jam. Therefore, we implement \textit{packet pair} to deal with this problem. With two packets sent at the same time, we are able to estimate the delay of traffic crowdedness by using the arrival time of two packets.

In the experiments, we found that our method XXXXXXXXXXXXXXXXXXXXXXXXXXXXXXXXXXXXXXXXX. Finally, we will discuss about the future expectation of this work.

The \emph{main contributions of this paper} are as follow:

\begin{enumerate}
\item
Analysis several types of attacks that influence the visibility of network.
\item
Discuss about counter measurements in previous works.
\item
Improve an existing switch entry validation method.
\item
Create an innovative fake link detection method.
\item
Evaluate and discuss about our methods.
\end{enumerate}

The following chapters in this thesis will be: Chapter 2 gives detail background knowledge of the used technology, discuss about possible threats and countermeasure. Chapter 3 is about our threat model and the theory of our own detection method. XXXXXXXXXXXXXXXXXXXXXXXXXXXXXXXXXXXXXXXXX.
-----------------------------------------------
Chapter 4 contains the experimental details including setup, considerations, simulations of attacks and evaluation methods. In Chapter 5, the proposed method will be evaluated under different conditions. Finally, the conclusion and future expectation of this work will be in Chapter 6.