% Note that the a4paper option is mainly intended so that authors in
% countries using A4 can easily print to A4 and see how their papers will
% look in print - the typesetting of the document will not typically be
% affected with changes in paper size (but the bottom and side margins will).
% Use the testflow package mentioned above to verify correct handling of
% both paper sizes by the user's LaTeX system.
%
% Also note that the "draftcls" or "draftclsnofoot", not "draft", option
% should be used if it is desired that the figures are to be displayed in
% draft mode.
%
\documentclass[a4paper,12pt]{report}

\linespread{1.5} \setlength{\parskip}{0.3cm}
% Add the compsoc option for Computer Society conferences.
%
% If IEEEtran.cls has not been installed into the LaTeX system files,
% manually specify the path to it like:
% \documentclass[conference]{../sty/IEEEtran}

\usepackage{amssymb}
\usepackage{amsmath}
\usepackage{amsthm}
\usepackage{latexsym}
\usepackage{algorithm}
%\usepackage{algorithmic}
\usepackage{algpseudocode}
\usepackage{graphicx} 
\usepackage{tabularx,array}
\usepackage{amsfonts}
\usepackage{amsmath}
\usepackage{epsfig}
\usepackage{verbatim}
%\usepackage{subfigure}
\usepackage{wallpaper}
\usepackage[skins]{tcolorbox}
\usepackage{color}
\CenterWallPaper{.35}{figures/ccu.jpg}


% Some very useful LaTeX packages include:
% (uncomment the ones you want to load)


% *** MISC UTILITY PACKAGES ***
%
%\usepackage{ifpdf}
% Heiko Oberdiek's ifpdf.sty is very useful if you need conditional
% compilation based on whether the output is pdf or dvi.
% usage:
% \ifpdf
%   % pdf code
% \else
%   % dvi code
% \fi
% The latest version of ifpdf.sty can be obtained from:
% http://www.ctan.org/tex-archive/macros/latex/contrib/oberdiek/
% Also, note that IEEEtran.cls V1.7 and later provides a builtin
% \ifCLASSINFOpdf conditional that works the same way.
% When switching from latex to pdflatex and vice-versa, the compiler may
% have to be run twice to clear warning/error messages.


% *** CITATION PACKAGES ***
%
\usepackage{cite}
\usepackage{ulem}
% cite.sty was written by Donald Arseneau
% V1.6 and later of IEEEtran pre-defines the format of the cite.sty package
% \cite{} output to follow that of IEEE. Loading the cite package will
% result in citation numbers being automatically sorted and properly
% "compressed/ranged". e.g., [1], [9], [2], [7], [5], [6] without using
% cite.sty will become [1], [2], [5]--[7], [9] using cite.sty. cite.sty's
% \cite will automatically add leading space, if needed. Use cite.sty's
% noadjust option (cite.sty V3.8 and later) if you want to turn this off.
% cite.sty is already installed on most LaTeX systems. Be sure and use
% version 4.0 (2003-05-27) and later if using hyperref.sty. cite.sty does
% not currently provide for hyperlinked citations.
% The latest version can be obtained at:
% http://www.ctan.org/tex-archive/macros/latex/contrib/cite/
% The documentation is contained in the cite.sty file itself.
\usepackage{amsmath}
\usepackage{longtable}
\newcommand{\tabincell}[2]{\begin{tabular}{@{}#1@{}}#2\end{tabular}}%放在导言区 



% *** GRAPHICS RELATED PACKAGES ***
%
%\ifCLASSINFOpdf
 %  \usepackage[pdftex]{graphicx}
  % declare the path(s) where your graphic files are
  % \graphicspath{{../pdf/}{../jpeg/}}
  % and their extensions so you won't have to specify these with
  % every instance of \includegraphics
  % \DeclareGraphicsExtensions{.pdf,.jpeg,.png}
%\else
  % or other class option (dvipsone, dvipdf, if not using dvips). graphicx
  % will default to the driver specified in the system graphics.cfg if no
  % driver is specified.
 %  \usepackage[dvips]{graphicx}
  % declare the path(s) where your graphic files are
  % \graphicspath{{../eps/}}
  % and their extensions so you won't have to specify these with
  % every instance of \includegraphics
  % \DeclareGraphicsExtensions{.eps}
%\fi
% graphicx was written by David Carlisle and Sebastian Rahtz. It is
% required if you want graphics, photos, etc. graphicx.sty is already
% installed on most LaTeX systems. The latest version and documentation can
% be obtained at:
% http://www.ctan.org/tex-archive/macros/latex/required/graphics/
% Another good source of documentation is "Using Imported Graphics in
% LaTeX2e" by Keith Reckdahl which can be found as epslatex.ps or
% epslatex.pdf at: http://www.ctan.org/tex-archive/info/
%
% latex, and pdflatex in dvi mode, support graphics in encapsulated
% postscript (.eps) format. pdflatex in pdf mode supports graphics
% in .pdf, .jpeg, .png and .mps (metapost) formats. Users should ensure
% that all non-photo figures use a vector format (.eps, .pdf, .mps) and
% not a bitmapped formats (.jpeg, .png). IEEE frowns on bitmapped formats
% which can result in "jaggedy"/blurry rendering of lines and letters as
% well as large increases in file sizes.
%
% You can find documentation about the pdfTeX application at:
% http://www.tug.org/applications/pdftex



\usepackage{float}

% *** MATH PACKAGES ***
%
%\usepackage[cmex10]{amsmath}
% A popular package from the American Mathematical Society that provides
% many useful and powerful commands for dealing with mathematics. If using
% it, be sure to load this package with the cmex10 option to ensure that
% only type 1 fonts will utilized at all point sizes. Without this option,
% it is possible that some math symbols, particularly those within
% footnotes, will be rendered in bitmap form which will result in a
% document that can not be IEEE Xplore compliant!
%
% Also, note that the amsmath package sets \interdisplaylinepenalty to 10000
% thus preventing page breaks from occurring within multiline equations. Use:
%\interdisplaylinepenalty=2500
% after loading amsmath to restore such page breaks as IEEEtran.cls normally
% does. amsmath.sty is already installed on most LaTeX systems. The latest
% version and documentation can be obtained at:
% http://www.ctan.org/tex-archive/macros/latex/required/amslatex/math/


% *** SPECIALIZED LIST PACKAGES ***
%
%\usepackage{algorithmic}
% algorithmic.sty was written by Peter Williams and Rogerio Brito.
% This package provides an algorithmic environment fo describing algorithms.
% You can use the algorithmic environment in-text or within a figure
% environment to provide for a floating algorithm. Do NOT use the algorithm
% floating environment provided by algorithm.sty (by the same authors) or
% algorithm2e.sty (by Christophe Fiorio) as IEEE does not use dedicated
% algorithm float types and packages that provide these will not provide
% correct IEEE style captions. The latest version and documentation of
% algorithmic.sty can be obtained at:
% http://www.ctan.org/tex-archive/macros/latex/contrib/algorithms/
% There is also a support site at:
% http://algorithms.berlios.de/index.html
% Also of interest may be the (relatively newer and more customizable)
% algorithmicx.sty package by Szasz Janos:
% http://www.ctan.org/tex-archive/macros/latex/contrib/algorithmicx/




% *** ALIGNMENT PACKAGES ***
%
%\usepackage{array}
% Frank Mittelbach's and David Carlisle's array.sty patches and improves
% the standard LaTeX2e array and tabular environments to provide better
% appearance and additional user controls. As the default LaTeX2e table
% generation code is lacking to the point of almost being broken with
% respect to the quality of the end results, all users are strongly
% advised to use an enhanced (at the very least that provided by array.sty)
% set of table tools. array.sty is already installed on most systems. The
% latest version and documentation can be obtained at:
% http://www.ctan.org/tex-archive/macros/latex/required/tools/


%\usepackage{mdwmath}
%\usepackage{mdwtab}
% Also highly recommended is Mark Wooding's extremely powerful MDW tools,
% especially mdwmath.sty and mdwtab.sty which are used to format equations
% and tables, respectively. The MDWtools set is already installed on most
% LaTeX systems. The lastest version and documentation is available at:
% http://www.ctan.org/tex-archive/macros/latex/contrib/mdwtools/


% IEEEtran contains the IEEEeqnarray family of commands that can be used to
% generate multiline equations as well as matrices, tables, etc., of high
% quality.

%\usepackage{eqparbox}
% Also of notable interest is Scott Pakin's eqparbox package for creating
% (automatically sized) equal width boxes - aka "natural width parboxes".
% Available at:
% http://www.ctan.org/tex-archive/macros/latex/contrib/eqparbox/

% *** SUBFIGURE PACKAGES ***
%\usepackage[tight,footnotesize]{subfigure}
% subfigure.sty was written by Steven Douglas Cochran. This package makes it
% easy to put subfigures in your figures. e.g., "Figure 1a and 1b". For IEEE
% work, it is a good idea to load it with the tight package option to reduce
% the amount of white space around the subfigures. subfigure.sty is already
% installed on most LaTeX systems. The latest version and documentation can
% be obtained at:
% http://www.ctan.org/tex-archive/obsolete/macros/latex/contrib/subfigure/
% subfigure.sty has been superceeded by subfig.sty.

%\usepackage[caption=false]{caption}
\usepackage[font=footnotesize]{subfig}
% subfig.sty, also written by Steven Douglas Cochran, is the modern
% replacement for subfigure.sty. However, subfig.sty requires and
% automatically loads Axel Sommerfeldt's caption.sty which will override
% IEEEtran.cls handling of captions and this will result in nonIEEE style
% figure/table captions. To prevent this problem, be sure and preload
% caption.sty with its "caption=false" package option. This is will preserve
% IEEEtran.cls handing of captions. Version 1.3 (2005/06/28) and later
% (recommended due to many improvements over 1.2) of subfig.sty supports
% the caption=false option directly:
%\usepackage[caption=false,font=footnotesize]{subfig}
%
% The latest version and documentation can be obtained at:
% http://www.ctan.org/tex-archive/macros/latex/contrib/subfig/
% The latest version and documentation of caption.sty can be obtained at:
% http://www.ctan.org/tex-archive/macros/latex/contrib/caption/

% *** FLOAT PACKAGES ***
%
%\usepackage{fixltx2e}
% fixltx2e, the successor to the earlier fix2col.sty, was written by
% Frank Mittelbach and David Carlisle. This package corrects a few problems
% in the LaTeX2e kernel, the most notable of which is that in current
% LaTeX2e releases, the ordering of single and double column floats is not
% guaranteed to be preserved. Thus, an unpatched LaTeX2e can allow a
% single column figure to be placed prior to an earlier double column
% figure. The latest version and documentation can be found at:
% http://www.ctan.org/tex-archive/macros/latex/base/



%\usepackage{stfloats}
% stfloats.sty was written by Sigitas Tolusis. This package gives LaTeX2e
% the ability to do double column floats at the bottom of the page as well
% as the top. (e.g., "\begin{figure*}[!b]" is not normally possible in
% LaTeX2e). It also provides a command:
%\fnbelowfloat
% to enable the placement of footnotes below bottom floats (the standard
% LaTeX2e kernel puts them above bottom floats). This is an invasive package
% which rewrites many portions of the LaTeX2e float routines. It may not work
% with other packages that modify the LaTeX2e float routines. The latest
% version and documentation can be obtained at:
% http://www.ctan.org/tex-archive/macros/latex/contrib/sttools/
% Documentation is contained in the stfloats.sty comments as well as in the
% presfull.pdf file. Do not use the stfloats baselinefloat ability as IEEE
% does not allow \baselineskip to stretch. Authors submitting work to the
% IEEE should note that IEEE rarely uses double column equations and
% that authors should try to avoid such use. Do not be tempted to use the
% cuted.sty or midfloat.sty packages (also by Sigitas Tolusis) as IEEE does
% not format its papers in such ways.

% *** PDF, URL AND HYPERLINK PACKAGES ***
%
\usepackage{url}
% url.sty was written by Donald Arseneau. It provides better support for
% handling and breaking URLs. url.sty is already installed on most LaTeX
% systems. The latest version can be obtained at:
% http://www.ctan.org/tex-archive/macros/latex/contrib/misc/
% Read the url.sty source comments for usage information. Basically,
% \url{my_url_here}.
%\usepackage{amssymb}
%\usepackage{amsmath}
%\usepackage{amsthm}
\usepackage{latexsym}
\usepackage{graphicx}
\usepackage{array,booktabs}
\usepackage{multicol}
\usepackage{multirow}
\usepackage{tabularx}
\usepackage{textcomp}


\makeatletter
\def\@normalsize{\@setsize\normalsize{12pt}\xpt\@xpt
\abovedisplayskip 10pt plus2pt minus5pt\belowdisplayskip
\abovedisplayskip \abovedisplayshortskip \z@
plus3pt\belowdisplayshortskip 6pt plus3pt
minus3pt\let\@listi\@listI}


\def\subsize{\@setsize\subsize{12pt}\xipt\@xipt}
\def\section{\@startsection {section}{1}{\z@}{24pt plus 2pt minus 2pt}
{12pt plus 2pt minus 2pt}{\large\bf}}
\def\subsection{\@startsection {subsection}{2}{\z@}{12pt plus 2pt minus 2pt}
{12pt plus 2pt minus 2pt}{\subsize\bf}} \makeatother


% *** Do not adjust lengths that control margins, column widths, etc. ***
% *** Do not use packages that alter fonts (such as pslatex).         ***
% There should be no need to do such things with IEEEtran.cls V1.6 and later.
% (Unless specifically asked to do so by the journal or conference you plan
% to submit to, of course. )


% correct bad hyphenation here
\hyphenation{op-tical net-works semi-conduc-tor}

%\usepackage{fontspec}   				%加這個就可以設定字體
%\usepackage{xeCJK}       				%讓中英文字體分開設置
%\setCJKmainfont{標楷體} 					%設定中文為系統上的字型,而英文不去更動,使用原TeX字型
%\XeTeXlinebreaklocale "zh"             	%這兩行一定要加,中文才能自動換行
%\XeTeXlinebreakskip = 0pt plus 1pt     	%這兩行一定要加,中文才能自動換行

\begin{document}
\bibliographystyle{plain}
%
% paper title
% can use linebreaks \\ within to get better formatting as desired


% author names and affiliations
% use a multiple column layout for up to three different
% affiliations
\title{\bf I havn't named my topic yet~~~~ }
\author{{\bf Student: Chiu, Yen-Chun}
\\ {\bf Advisor: Prof. Lin, Po-Ching}
\\ National Chung Cheng University
\\ Ming-Hsiung, Chiayi 621, Taiwan
\\ shuaichiou@gmail.com}
\date{}

% conference papers do not typically use \thanks and this command
% is locked out in conference mode. If really needed, such as for
% the acknowledgment of grants, issue a \IEEEoverridecommandlockouts
% after \documentclass

% for over three affiliations, or if they all won't fit within the width
% of the page, use this alternative format:
%
%\author{\IEEEauthorblockN{Michael Shell\IEEEauthorrefmark{1},
%Homer Simpson\IEEEauthorrefmark{2},
%James Kirk\IEEEauthorrefmark{3},
%Montgomery Scott\IEEEauthorrefmark{3} and
%Eldon Tyrell\IEEEauthorrefmark{4}}
%\IEEEauthorblockA{\IEEEauthorrefmark{1}School of Electrical and Computer Engineering\\
%Georgia Institute of Technology,
%Atlanta, Georgia 30332--0250\\ Email: see http://www.michaelshell.org/contact.html}
%\IEEEauthorblockA{\IEEEauthorrefmark{2}Twentieth Century Fox, Springfield, USA\\
%Email: homer@thesimpsons.com}
%\IEEEauthorblockA{\IEEEauthorrefmark{3}Starfleet Academy, San Francisco, California 96678-2391\\
%Telephone: (800) 555--1212, Fax: (888) 555--1212}
%\IEEEauthorblockA{\IEEEauthorrefmark{4}Tyrell Inc., 123 Replicant Street, Los Angeles, California 90210--4321}}




% use for special paper notices
%\IEEEspecialpapernotice{(Invited Paper)}




% make the title area
\maketitle

\setcounter{page}{1}
\pagenumbering{roman}
\begin{abstract}
\label{sec:abstract}
	Software defined network (SDN) is a next-generation concept that allows network administrator to manage network flows with ease. It is programmable, centrally managed, and being able to adapt to the change of topology. Meanwhile, these characteristics also leads to new security problems, which some have already been discussed and proven. And people also give out great protection mechanisms and suggestions. However, there are still more possibilities of attacks in different scenario left to be discovered. And as SDN evolves, it will definitely cause some more new issues. In this paper, we analsis several types of topology-related attacks, and create a method that is able to detect topology poisoning attacks by using the standard functionality of OpenFlow switch. We evaluate the effectiveness of our method under different conditions, and discuss the possibility of future works.
\end{abstract}
%backup=====================
%According to the profiling, we find pattern matching can dominate the execution time if the NIDS is configured to scan every packet payload in the packet traces.
%Moreover, the preprocessing stage can be heavy when the NIDS handles packet reassembly for IP segments in malicious traffic.
%Adjusting the depth of payload to be scanned in the system configuration will be effective for performance adaptation under heavy load.
%============================
% IEEEtran.cls defaults to using nonbold math in the Abstract.
% This preserves the distinction between vectors and scalars. However,
% if the conference you are submitting to favors bold math in the abstract,
% then you can use LaTeX's standard command \boldmath at the very start
% of the abstract to achieve this. Many IEEE journals/conferences frown on
% math in the abstract anyway.

%\begin{keywords}
%NIDS, adversarial network traffic, system performance, pattern matching.
%\end{keywords}

% For peer review papers, you can put extra information on the cover
% page as needed:
% \ifCLASSOPTIONpeerreview
% \begin{center} \bfseries EDICS Category: 3-BBND \end{center}
% \fi
%
% For peerreview papers, this IEEEtran command inserts a page break and
% creates the second title. It will be ignored for other modes.
%\IEEEpeerreviewmaketitle

\tableofcontents\listoffigures\listoftables
\chapter{Introduction}
\label{chap:intro}
\setcounter{page}{1}
\pagenumbering{arabic}

Nowadays, as the popularity of public clouds like Google cloud, Microsoft Azure, Amazon EC2 increase, we can see how cloud computing offer a new way of deploying applications and services. While people try to move everything to the cloud, the growth of volume and complexity of data center network (DCN) are getting out of hand, and server virtualization is also becoming more and more common. Networking organization are under increasing pressure to be more efficient, agile, and maintainable than is possible with the traditional approach to networking.

In traditional network, most of the network functionality heavily rely on hardware because they are implemented in network devices, which are under the control of manufacturers. As a result, the evolution of those functionality will be limited. Furthermore, implementing a network-wide policy requires configuring at the device-level. And similarly, tasks such as provisioning, change management, and de-provisioning are also very time-consuming, error-prone and require a lot of manpower. With the scalability problem, it is surely a formidable job for network administrators to manage a large scale network with traditional network. 

Software defined network (SDN) is a dynamic, manageable, cost-effective, and adaptable network network structure. It gains great popularity among enterprises as well as academia recent years. SDN separates the control plane from data plane, it is centrally controllable by software applications. Comparing to legacy network, it is more flexible, maintainable and agile. Also, components of legacy network such as a regular switch is totally compatible within a SDN network structure.

Nevertheless, new technology often comes with new security problems. In addition to switches and hosts, SDN uses a controller to realize centralized control, and there are applications in the controller. There are also some new mechanisms such as topology discovery, host managing, protocols and API for the communication of entities. With so many new elements introduced, there will be more potential security issues that need to be taken care of in different ways. We will talk more about these issues in later chapters.

\emph{The motivation of our work} is that, although lots of works has been done to deal with all these security problems in SDN, some aspect are not included in the works. For example, in \cite{HXWG_15}, Hong et al. propose attacks that poison network visibility and its countermeasure, but did not consider the situation that switches are compromised. And to the best of our knowledge, although some protection method are proposed, there has not been an effective way to detect if there is any switch being compromised in the network. 

With the program-configurable and manageable traits of SDN, we believe it is possible to implement similar defensive solution with the aids of those SDN properties. \emph{The main goal of this thesis} is to promotes existing method and try to propose a new detection method using the standard functionality of OpenFlow switch. With this work, we hope we are able to set an example to inspire others and draw more attention from the community to concern about the security issues in SDN, rise the security awareness of SDN users, and ultimately resulting in a more mature SDN environment.

During the research, we study the specification of components as well as the potential threats in SDN. After discussing about those attacks, we try to cover the situations that were left by the works of the others. Then some threat models and the frame of our innovative method is proposed hypothetically. 
In the experiments, we found that XXXXXXXXXXXXXXXXXXXXXXXXXXXXXXXXXXXXXXXXX. Finally, we will assess how effective our method is.

The \emph{main contributions of this paper} are as follow:

\begin{enumerate}
\item
Analysis several types of attacks that influence the visibility of network.
\item
Discusses about previously-created counter measurements.
\item
Promote a currently-existing switch entry validation method.
\item
Create an innovative detection measure and improve currently-existing method.
\item
Evaluate our method under different conditions.
\end{enumerate}

The following chapters in this thesis will be: Chapter 2 gives detail background knowledge of the used technology, discuss about possible threats and countermeasure. Chapter 3 is about our threat model and the theory of our own detection method. XXXXXXXXXXXXXXXXXXXXXXXXXXXXXXXXXXXXXXXXX.
-----------------------------------------------
Chapter 4 contains the experimental details including setup, considerations, simulations of attacks and evaluation methods. In Chapter 5, the proposed method will be evaluated under different conditions. Finally, the conclusion and future expectation of this work will be in Chapter 6.
\chapter{Background and related work}
\section{SDN and OpenFlow}
OpenFlow is the most popular southbound interface in SDN. A switch that supports OpenFlow is called an OpenFlow switch. Aside from physical switches, there are also software implementations of virtual switches, such as \textit{Open vSwtich} (\url{openvswitch.org}). OpenFlow switches typically separate OpenFlow and non-OpenFlow traffic, which do not interfere with each other \cite{HP_SPEC}.

A controller is able to determine the forwarding path of packets by adding, updating and deleting flow entries in the flow tables of OpenFlow switches in both reactive and proactive ways \cite{OF_SPEC}. It also maintains the abstract view of the network, including network topology, host positions and the states of network resources. An incoming packet from the \textit{ingress port} will go through one or more flow tables, optionally through the group table, and be processed according to the actions defined by the matched flow entry. Each OpenFlow table typically contains thousands of flow entries. Figure~\ref{FE_Col} presents the columns of a flow entry. Packets will be matched with the \textit{match fields} of a flow entry. When a packet matches a flow entry, the action set to that packet will be modified according to the instructions in the entry. There is an entry with the lowest priority that matches all fields. It is for packets that cannot match any other flow entries. Normally, such a packet will be encapsulated and sent to the controller, which will decide how to process it and add a new flow entry according to the network policy. After the end of processing pipeline, the actions in the action set will be executed.% When ports are added or removed, the content of flow tables remain unchanged, so the controller should clean up the reference of a port if a port is deleted \cite{OF_SPEC}.

\begin{figure}[H]
\begin{center} 
\includegraphics[width=1\textwidth]{figures/columns_of_flow_entry.png}
\end{center}
\caption{Columns of a flow entry.}
\label{FE_Col}
\end{figure}

\section{Topology discovery services}
\label{Topology discovery services}
The controller needs to maintain the visibility of the whole network to realize centralized control and high programmability. Therefore, topology discovery services play an important role in SDN. The services help to reduce manual efforts significantly when the topology is altered. The topology services include three parts: switch discovery, host tracking and internal link.

\subsection{Switch discovery service and Host tracking service}
Switch discovery is rather simple. When a switch initiates a connection with a controller, the OpenFlow channel will be established, and the switch information will be sent to the controller. A controller maintains the host profiles to keep track of the locations of hosts. When a packet-miss happens in the flow table, a \texttt{Packet\_In} message will be sent to the controller along with the packet's information and the controller will look up the host profiles it maintains. If the host profile of the host cannot be found, the controller will assume a new host joining the network and add the information of the host. If there is a conflict between the host profile and the \texttt{Packet\_In} message, the controller treats this as a host migration and updates the location in the host profile.

\subsection{Link discovery service}
\label{Link discovery service}
Link discovery refers to the procedure of discovering the links between switches. Since there has not been a standard for the link discovery in the OpenFlow controller, we will use the term \textit{OpenFlow Discovery Protocol} (OFDP) when mentioning it. Currently, all mainstream controllers support OFDP despite some minor differences in detail.

OFDP leverages the Link Layer Discovery Protocol (LLDP) with subtle modification to perform topology discovery in an OpenFlow network. LLDP is originally implemented for an Ethernet switch to exchange its identity and capabilities with adjacent layer-2 peers. In a legacy network, LLDP packets are sent regularly via each port of switches \cite{LLDP_WS}. A switch stores the information learned from LLDP packets sent by the neighbors and the packets will not be forwarded further. Figure~\ref{LLDP_frame} shows the structure of an LLDP Ethernet frame. Each LLDP data unit contains a sequence of type-length-values (TLV). 

\begin{figure}[H]
\begin{center} 
\includegraphics[width=1\textwidth]{figures/LLDP_packet_format.png}
\end{center}
\caption{LLDP packet frame structure. \cite{LLDP_WS}}
\label{LLDP_frame}
\end{figure}

However, OFDP operates quite differently. The controller keeps the topology information, and an OpenFlow switch does nothing more than forwarding the LLDP packet. The simplified process is shown in Figure~\ref{OFDP}. All switches have a pre-installed rule in their flow tables. The rule specifies to send LLDP packets received from any ports except the controller port back to controller via \texttt{Packet\_In}. Initially, the controller creates an LLDP packet for each port on every switch via the \texttt{Packet\_Out} message. After receiving the LLDP packet from controller, S1 sends it out on Port 1 and received by S2 on Port 3. With the pre-installed forwarding rule, switch S2 forwards the received LLDP packet to the controller via a \texttt{Packet\_In} message, which contains meta-data such as the identifier of the switch and the ingress port via which the packet was received. Thus, the controller can now infer that there exists a link between Port 1 of S1 and Port 3 of S2, and this information will be added to controller's topology database. After running this process through all the ports on all the switches, the controller will obtain all links between switches in the network. The entire discovery process is performed periodically with a typical default interval size of 5 seconds \cite{PPTI14}. 

\begin{figure}[H]
\begin{center} 
\includegraphics[width=1\textwidth]{figures/OFDP_procedure.png}
\end{center}
\caption{An illustration of OFDP procedure.}
\label{OFDP}
\end{figure}

\section{SDN security}
\label{SDN security}
A thorough review of SDN-related security issues can be found in \cite{LAB14, CM, SOS13, KJK}. Therefore, rather than repeat the review, we will focus on the issue of compromising OpenFlow switches, since it has been less addressed than others so far. Compromising OpenFlow switches can lead to some bad results. (1) Attacker may launch topology poisoning attack by manipulating link discovery packet. (2) The actions specified in the flow rules on a compromised switch can be unexpectedly altered. (3) The packets through a compromised switch can be eavesdropped or dropped. (4) The compromised switch may be configured to be managed by another malicious controller. (5) It is possible to launch network-wide DOS attack by sending specific forged packet to consume the controller's resource.

The main idea of the topology poisoning attack is to trick the controller into believing the existence of a non-existing link to host or switch by exploiting traits of topology management service. One can initiate such type of attack with either a switch or a host. In \cite{HXWG15}, Hong et al. mentioned Host Location Hijacking Attack and Link Fabrication Attack, and presented TopoGuard to solve the problem. However, LLDP packets are passed around with the aid of switches. The Link Fabrication Attack can be also initiated by compromised switches, but the attack vector  is not covered in the scenario of TopoGuard. Bui gives three different attack scenarios of Link Fabrication Attack with compromised switches and evaluates their consequence under different routing algorithms and network topologies \cite{TTB15}. This attack is caused by the lack of authentication of LLDP. However, simply adding authenticator inside the LLDP packet will not help against LLDP relay attack \cite{HXWG15}. Alharbi et al. implement HMAC based mechanism with a little modification to static secret key, which is able to detect the injection of any fabricated LLDP packets, with only an acceptable of amount of overhead added \cite{ATPP15}.

Attackers can also modify the flow entries inside the flow tables of the compromised switch to perform MITM, eavesdropping or DOS attack \cite{AAS14}. The detection method proposed in \cite{CKGL15} is able to detect whether a switch is forwarding the packets in an unexpected way. After selecting a flow entry as the detecting target, they install new entry on its neighbors. With the match field selected by their algorithm, they are able to let every packet that matches the new flow entry matches the target flow entry. A packet containing the match field of the new flow entry will be sent from \texttt{Packet\_Out} to a neighbor of the target switch, forwarded to the target switch, and should be sent back to the controller. Finally, they will check if the packet comes back to the controller as expected and remain unchanged. However, this method will take a long time to run if it is desired to scan through a large number of flow entries. A pre-detection method to narrow down the potential target is needed.
\chapter{Flow entry verification and Fake link detection}
This work focuses on the problem that a compromised switch will bring and presents the methods to detect such a switch. This work involves two parts: \textit{flow entry verification} and \textit{fake link detection}. We will define the threat model that both parts share, and explain them along with their attack scenarios and detection algorithms.

\section{Threat model and Attack scenario}
In this work, we assume the following scenario of compromising a switch in the threat model:
\begin{enumerate}
\item
One controller is used, and only one OpenFlow switch is compromised. No cooperation among multiple compromised switches for attacks will happen.
\item
The switches are able to access the Internet. 
\item
Other parts of the network such as the controller, other switches and hosts function normally. Any potential flaw is unintentional and is out of the scope of this work.
\item
An attacker cannot totally change the way of switch processing or core mechanism, but only perform the attack by modifying flow entries or forge LLDP to cause fake links.
\item
Initially, the network is all clean, and nothing is compromised. The attacks take place some time after the whole network is established.
\end{enumerate}

The attack scenarios beyond the assumptions such as multiple compromised switches, as well as possible circumvention, will be discussed in Section~\ref{Further_discussion}.

\section{Flow entry verification}
In the first part of this work, we propose a method to detect if the flow entries of a switch work as expected. The method is inspired by \cite{CKGL15}. The method in this work has two main enhancements. First, it reduces the number of detection packets required, and therefore increases the efficiency significantly. Second, the additional flow entries we need to install is less than the previous method, resulting in lower cost for setting up and cleaning up. 

\subsection{Detection method}
\label{Detection_method}
The main idea of the method is to assemble a packet that will go through a series of switches by matching the match fields in the flow entries of those switches. Then the packet will be sent into the network, and should be sent back to the controller finally if nothing goes wrong. For this purpose, we take advantage of the network-wise visibility provided by the controller. The flow chart of the flow entry detection process in the controller is shown in Figure~\ref{flow_entry_detection_flowchart}. 

\begin{figure}[H]
\begin{center} 
\includegraphics[width=1\textwidth]{figures/flow_entry_detection_flowchart.png}
\end{center}
\caption{The flow chart of the flow entry detection process.}
\label{flow_entry_detection_flowchart}
\end{figure}

First, we define the \textit{aggregated groups}. An aggregated group consists of one or more than one flow entries in the switches. Each of the flow entries has an output action pointing to the next switch which contains a flow entry that is also in the same aggregated group. In addition, the flow entries in the same aggregated group must satisfy either of the two conditions below \red{but you list three below}, and we name them the \textit{aggregation condition} for later usage: 
\begin{enumerate}
\item
There is at least one other flow entry that has exactly the same match field and same value in a different switch.
\item
There is no other flow entry with the same match field.
\item
Entries in the same group must not have forward action to a switch that cause cycle.
\end{enumerate}

With the complete view of network provided by the controller and the aggregation condition, the controller is able to construct a detection packet to be sent from the first flow entry all the way down to the last flow entry in the same aggregated group, and finally come back to the controller, indicating that all the flow entries inside this aggregate group work as expected. 

Figure~\ref{aggregated_group} illustrates an aggregated group and the content of a constructed packet. The flow entries connected by arrows belong to the same group. After determining an aggregated group, the controller constructs a packet according to the match fields of flow entries inside the aggregated group. With the aggregation condition, it will match every flow entry inside the aggregated group. In the example figure, we have a controller, three switches, which are S1,S2 and S3, and a packet waiting to be constructed. In S1, there is a flow entry that contains match field eth\_dst, match value A and output action to S2. In S2, the flow entry has match field ipv4\_src, match value B and output action to S3. Initially in S3, there is no flow entry that matches the aggregation condition of this aggregated group, so we use it as the ending node and set a flow entry with arbitrary match field (in the example, arp\_op is used) and value C in order to notify the controller. According to these match fields, the constructed packet will contain eth\_dst as A, ipv4\_src as B and arp\_op as C. Then the packet is sent to S1, and will match to the flow entry and be sent to S2. After this, the packet will match the second flow entry in the aggregated group and be sent to S3. Finally, when it reaches the last flow entry made by us, it will be sent back to the controller, and the processing of this aggregated group is done.

\begin{figure}[H]
\begin{center}
\includegraphics[width=1\textwidth]{figures/aggregated_group.png}
\end{center}
\caption{An aggregated group and constructed packet.}
\label{aggregated_group}
\end{figure}

The process of generating a packet is done along with defining one aggregated group, the pseudo-code of it is as follow:

\begin {tcolorbox}[blanker,float=tbp,
grow to left by=1cm, grow to right by=1cm]
\begin{algorithm}[H]

  \caption{Packet generating process.}
  \begin{algorithmic}[1]
    \Require
      Information of all switches, $switches$;
      Global visited status of every flow entry, $visited\_entry$;
      Visited status of every switch in a aggregation group, $visited\_switch$
      An under-constructing packet that is initially empty, $packet$; 

    \Function{flow\_entry\_traversal}{switches}
      \ForAll{$switch$ in $switches$}
        \ForAll{$entry$ in $switch$}
          \State Extract $match\_field$, $match\_value$, $action$, $cookie$ from $entry$;
          \State $entry\_id \gets \textit{cookie}$; //use cookie as entry identifier
          \State 
          \State Extract $next\_switch$ from $action$;
          \If{NOT $visited\_entry$[$entry\_id$] and NOT $visited\_switch$[$next\_switch$]}
            %\State $\textit{visited\_switch}[\textit{switch}] \gets true$;
            \State $\textit{visited\_entry}[\textit{entry\_id}] \gets true$;
            \State $packet[\textit{match\_field}] \gets \textit{match\_value}$;
            \State $\textit{complete\_packet} \gets \Call{packet\_gen}{\textit{match\_field}, \textit{match\_value}, \textit{action}, \textit{packet}}$;
            \State Send $\textit{complete\_packet}$ with PACKET\_OUT;
          \EndIf
      \EndFor
        \EndFor
    \EndFunction
    \State
    \Function{packet\_gen}{match\_field$, match\_value$, action$, packet$}      
      \ForAll {$entry$ in $next\_switch$}
        \State Extract $match\_field$, $match\_value$, $action$, $cookie$ from $entry$;
        \State Extract $next\_switch$ from $action$; 
  \algstore{packet_generating}
  \end{algorithmic}
\end{algorithm}
\end{tcolorbox}

\begin {tcolorbox}[blanker,float=tbp,
grow to left by=1cm, grow to right by=1cm]
\begin{algorithm}[H]
  \begin{algorithmic}[1]
  \algrestore{packet_generating}
        \State $\textit{entry\_id} \gets \textit{cookie}$;
        \If{NOT $visited\_entry$[$entry\_id$] and NOT $visited\_switch$[$next\_switch$]}  
          \State $\textit{visited\_entry}[\textit{entry\_id}] \gets true$;
          \State $\textit{visited\_switch}[\textit{next\_switch}] \gets true$;
          \If{$packet$[$match\_field$] is set and $packet$[$match\_field$] = $match\_value$}            
            \State \Return \Call{packet\_gen}{$match\_field$, $match\_value$, $action$, $packet$};
          \ElsIf{$packet$[$match\_field$] is not set}
            \State $\textit{packet}[\textit{match\_field}] \gets \textit{match\_value}$;
            \State \Return \Call{packet\_gen}{$match\_field$, $match\_value$, $action$, $packet$};
          \EndIf
        \EndIf
      \EndFor
      \State Install new flow entry with action forwarding to controller into $next\_switch$;
      \State $\textit{visited\_switch} \gets EMPTY$;
      \State \Return $packet$;
    \EndFunction
  \end{algorithmic}
\end{algorithm}
\end{tcolorbox}

In order to optimize the number of necessary detection packets, the whole network should be covered with the minimum number of aggregated groups. By taking switches as vertex and forwarding actions as edges, it forms a complex graph problem. It is more complex than Longest path problem, which is also np-complete in time complexity. Inspired by Euler Path, we calculate the number that every vertex is destination of some edges, and use it as visiting priority as a approximation method. The reason behind the thought is that, the vertex who is treated as the destination the least time should be used as start point in order to reduce the number of necessary groups.

The cookie is used to identify every unique entry in different switches. When defining an aggregated group, we start with an unvisited flow entry that currently is treated as destination the least time, seek for an unvisited flow entry that matches the aggregation condition, and go to the next flow entry in other switch according to the output action recursively. Eventually, switch that does not have any entry that matches the aggregation condition for this group is reached, the aggregated group defining process ends, and a packet will be generated in correspondent to the flow entries inside the aggregated group. According to the aggregation condition, every aggregated group will be mutual exclusive. By going through through all the flow entries inside the switches, we should be able to cover all of the flow entries with multiple aggregated group. After all of the flow entries with output action is visited and belong to certain aggregated groups, and all the flow entries will be verified after sending the same number of packets with the number of aggregated groups. 

\subsection{Additional consideration of flow entry verification method}
\label{Further_discussion}

In real case, the variety of the conditions aggregation may be much more complicated. A flow entry might have multiple match fields. Multiple flow table may also cause similar problem. When the action in a flow entry send the packet to next table for further processing. The packet is also matched against more than one match field. These kind of situations can still be covered by our method with some modification. 
If all the fields and values match the under-constructing packet, we process it according to first aggregation rule. Otherwise we fill the packet with all of those match fields. If any match fields is already taken with different value, it will be ignored for this turn of aggregation process and fill the match fields in another packet that has no conflict field. However, the number of flow entries aggregated in a packet is reduced, and so is the effectiveness. Another condition is the conjunction action that ties groups of individual flows with same match field and different value into ``conjunctive flows'' and reduce the number of flow entry \cite{OVS_OFCTL}. It can also be dealt in a specific way. However, the main purpose of our method is to show the effectiveness of flow entry aggregation method. We will only demonstrate with the most simple condition, which is a match field that has one value.

It is certainly possible that more than one switch are made by the same manufacturer or have the same software version in the same network, so multiple switches share the same vulnerabilities and may be compromised simultaneously. However, cooperation between multiple compromised switches complicates the scenario significantly. Taking our own method as an example, suppose there are two compromised switches, CS1 and CS2, which are neighbors to each other, and both contain an entry in the same aggregation group. When CS1 receives the detection packet and forwards it to other place maliciously, it sends one copy of detection packet to CS2, the other copy back to the controller, and the rest of the detection process will continue without raising any suspicion\sout{\red{Since the controller does not receive the detection packet, is it possible to raise a suspicion here?}}. In order to simplify the case as a starting point for developing the detection method, we assume only one switch inside the network is compromised.

Suppose an attacker is able to add, remove, or modify the entries in the flow tables of a compromised
switch without notifying the controller, so packets may be forwarded to an undesired destination. The proposed method is intended to detect this behavior with high efficiency. In this method, only output action of packets is considered. Other actions such as dropping packets, setting field, changing TTL will be ignored \sout{\red{Cann't they (e.g., dropping packets) be detected?}}. Although there are reasonable methods to deal with these actions, for example, to detect packet dropping, we can add timeout checking function, such detection function is irrelevant to our main method and is not implemented in our work.

In this detection method, there is another important consideration: the controller must maintain unpolluted information of all flow entries. It is not reliable for a controller to obtain flow entry information by querying switches because a switch may be malicious and forge a fake response that gives false information about the flow entries. It is also possible to initiate OpenFlow message such as adding a new flow entry, resulting in adding a malicious entry without raising any alarm. However, it also means that it makes this malicious behavior known to controller and is quite noticeable.

\section{Fake link detection}

-----
When a switch is compromised, an attacker is surely able to drop LLDP and causes DOS. However, it is very noticeable. We will focus on the MITM types of attack
-----
\subsection{Attack scenario}
\subsection{Detection method}


-----------------------------------------------------------

Below is our attack scenario assumption: 
- The control channel is properly protected with TLS protocol, meaning
that it provides confidentiality for the control traffic as well as mutual
authentication between the controller and switches.

Packet pair

- The dispersion in normal vs attack scenario, regardless of congestion

----
If this is not the case, further narrow down process is needed to find the location of the compromised switch XXXXXXXXXXXXXXXXXXXXXXX.
----

Time and space complexity


Packet forwarded to non-existent ports are just dropped

\chapter{Detection Method Implementaion and Evaluation}
In this chapter, we detail experimental detail of methods we proposed in chapter 3. First, the tools and config settings will be listed. Then, we will 

\section{General experimental setup}
We use Mininet as our network environment. It allows us to creates a large-scale virtual network easily. 
It provides python API as well as command line interface to customize the network. It also offers an interactive interface to test the connectivity and performance. We use the virtual machine provided by the OpenFlow tutorial \cite{OFT}. It comes with all the elements we need, such as Mininet, ryu controller, OpenVswitch etc.

Table~~\ref{table:Experiment_table} is a summary of all the tools used. 

In both of our methods, hosts are irrevelant, so there will not be any host in our environment. Some setup like topology or number of switches are different in two parts of our work, they will be described individually.

\begin{table}[H]
\centering
\caption{Experimental environment summary}
\begin{tabular}{|l|p{4cm}|p{4.5cm}}
\hline Item & Detail version \\
\hline Operating system & Ubuntu 12.10 x86\_64 \\
\hline Controller & ryu\_manager 3.6 \\
\hline Network Emulator & Mininet 2.1.0\\
\hline Southbound API & OpenFlow 1.0.0 \\
\hline Virtual switch & OpenvSwitch 1.11.0 \\
\hline 
\end{tabular}
\label{table:Experiment_table}
\end{table}


\section{Method Implementation}
preinstall flow entries proactively to simulate the . 
\subsection{}



\section{Evaluation}


there will be only \textit{output to port} actions among all the flow entries of all the switches in our environment. \ref{flow_entry_verification_attack_scenario} 
\chapter{Conclusion and future work}
\label{conclusion}
In this work, we discuss the hazard that an compromised switch in SDN network may bring. In the scenario we set up, we can see how the problems can be solved by using SDN features. Even the scenario is different, we believe that the new problem it introduced can also be solved with some extension and adjustment with the functionality SDN provides. We propose an effective way that is able to scan through the entire network for disobedient forwarding behavior with less number of packets, and evaluate it under different network topology type, scale and entry number on each switch.

The experimental result demonstrates that our method is effective under a network topology with balance structure. Also, the scale of the network topology does not effect the effeciency significantly, and the aggregation rate grows to 3.48 in fat tree topology once there are 120 entries in each switch. 

In the future work, more realistic factors should be take into consideration, such as the collaboration of multiple compromised switches, the entries with multiple match fields and wildcard fields. Furthermore, the aggregated group finding method currently performs DFS without any estimation, the core algorithm can be improve by adding heuristic function. Also, as the new versions of OpenFlow release, the method should able to be optimized with the support of the new features. For example, the auxiliary entries can be installed on egress table once it is fully supported.
\begin{thebibliography}{1}

\bibitem{NH12}
Adnan Nadeem and Michael P. Howarth,
``A Survey of MANET Intrusion Detection \& Prevention Approaches for Network Layer Attacks,'', In Proc. of IEEE COMMUNICATIONS SURVEYS \& TUTORIALS, VOL. 15, NO. 4, FOURTH QUARTER 2013.

\bibitem{GKPPCMS08}
Natasha Gude, Teemu Koponen, Justin Pettit, Ben Pfaff, Martín Casado, Nick McKeown and Scott Shenker
``NOX: Towards an Operating System for Networks'', ACM SIGCOMM Computer Communication Review 38.3 (2008): 105-110.

\bibitem{EZA11}
EugeneNg, ZhengCai AlanL Cox TS.,
``Maestro: Balancing fairness, latency and throughput in the openflow control plane'', Tech. rep., Rice University, 2011.

\bibitem{KJK}
Anurag Khandelwal, Navendu Jain and Seny Kamara
``Attacking Data Center Networks from the Inside'', 

\bibitem{AAS14}
Markku Antikainen, Tuomas Aura, and Mikko Särelä,
``Spook in Your Network: Attacking an SDN with a Compromised OpenFlow Switch'', In proc. of Springer International Publishing Switzerland 2014.

\bibitem{LNRSW04}
T.V. Lakshman, T. Nandagopal, R. Ramjee, K. Sabnani and T. Woo
``The softrouter architecture'', In Proc. of ACM SIGCOMM Workshop on Hot Topics in Networking. Vol. 2004. 2004.

\bibitem{TTB15}
TIEN THANH BUI,
``Analysis of Topology Poisoning Attacks in Software-Defined Networking'', Degree project in security and mobile computing, Second Level Stockholm, Sweden 2015.

\bibitem{PPTI14}
Farzaneh Pakzad, Marius Portmann, Wee Lum Tan and Jadwiga Indulska,
``Efficient Topology Discovery in Software Defined Networks'', In Signal Processing and Communication Systems (ICSPCS), 2014 8th International Conference on, pp. 1-8. IEEE, 2014. 

\bibitem{CKGL15}
Po-Wen Chi, Chien-Ting Kuo, Jing-Wei Guo and Chin-Laung Lei,
``How to Detect a Compromised SDN Switch'', In Network Softwarization (NetSoft), 2015 1st IEEE Conference on (pp. 1-6). IEEE.

\bibitem{HXWG15}
Sungmin Hong, Lei Xu, Haopei Wang and Guofei Gu,
``Poisoning Network Visibility in Software-Defined Networks: New Attacks and Countermeasures'', In NDSS. 2015. 

\bibitem{ATPP15}
Alharbi, Talal, Marius Portmann, and Farzaneh Pakzad,
``The (In) Security of Topology Discovery in Software Defined Networks.'', In Proc. of Local Computer Networks (LCN), 2015 IEEE 40th Conference on. IEEE, 2015.

\bibitem{KDFRV13}
Kreutz, Diego, Fernando Ramos, and Paulo Verissimo, 
``Towards secure and dependable software-defined networks'', In Proc. of the second ACM SIGCOMM workshop on Hot topics in software defined networking. ACM, 2013.


\bibitem{FFR09}
Ferraiolo, David F., and D. Richard Kuhn., 
``Role-based access controls'', arXiv preprint arXiv:0903.2171 (2009).

\bibitem{}
Hu, Ningning, and Peter Steenkiste,
``Estimating available bandwidth using packet pair probing'', No. CMU-CS-02-166. CARNEGIE-MELLON UNIV PITTSBURGH PA SCHOOL OF COMPUTER SCIENCE, 2002.

\bibitem{HRKC12}
Ralph Holz, Thomas Riedmaier, Nils Kammenhuber, Georg Carle, 
``X. 509 Forensics: Detecting and Localising the SSL/TLS Men-in-the-middle'' , In Computer security–esorics 2012 (pp. 217-234). Springer Berlin Heidelberg.

\bibitem{YZP11}
Yan, Zheng, and Christian Prehofer,
``Autonomic trust management for a component-based software system'', Dependable and Secure Computing, IEEE Transactions on 8.6 (2011)

\bibitem{SPYFGT13}
Shin, S., Porras, P. A., Yegneswaran, V., Fong, M. W., Gu, G., and Tyson, M.,
``FRESCO: Modular Composable Security Services for Software-Defined Networks'', in NDSS 2013.

\bibitem{PSYFTG12}
Phillip Porras, Seungwon Shin, Vinod Yegneswaran, Martin Fong, Mabry Tyson and Guofei Gu,
``A Security Enforcement Kernel for OpenFLow Networks'', In Proc. of the first workshop on Hot topics in software defined networks. ACM, 2012.

\bibitem{OF_SPEC}
\url{https://www.opennetworking.org/images/stories/downloads/sdn-resources/onfspecifications/openflow/openflow-switch-v1.5.1.pdf}

\bibitem{OFDP_GENI}
\url{http://groups.geni.net/geni/wiki/OpenFlowDiscoveryProtocol}

\bibitem{LLDP_WS}
\url{https://wiki.wireshark.org/LinkLayerDiscoveryProtocol}

\bibitem{SSAV_NWW}
\url{http://www.networkworld.com/article/2840273/sdn/sdn-security-attack-vectors-and-sdn-hardening.html}

\bibitem{TTN}
\url{http://www.wipro.com/documents/insights/the-thinking-network.pdf}

\bibitem{HP_SPEC}
\url{http://h10032.www1.hp.com/ctg/Manual/c04495114}

\bibitem{SHELLSHOCK}
\url{https://web.nvd.nist.gov/view/vuln/detail?vulnId=CVE-2014-6271}

\bibitem{POODLE}
\url{https://web.nvd.nist.gov/view/vuln/detail?vulnId=CVE-2014-3566}

\bibitem{HB}
\url{https://web.nvd.nist.gov/view/vuln/detail?vulnId=CVE-2014-0160}

\bibitem{PP}
\url{http://usenix.org/legacy/publications/library/proceedings/usits01/full_papers/lai/lai_html/node2.html}
----------------

\bibitem{ASO08}
T. T. Nguyen and G. Armitage, 
``A Survey Of Techniques for Internet Traffic Classification Using Machine Learning,'' In Proc. of IEEE Comm. Surveys Tutorials, vol. 10, no. 4, pp. 56-76, Oct.-Dec.2008.

\bibitem{PIO13}
J. S. Park, S. H. Yoon and M. S. Kim,
``Performance Improvement of Payload Signature-based Traffic Classification System Using Application Traffic Temporal Locality,'' In Proc. of the 15th Network Operations and Management Symposium (APNOMS), Sept. 2013. 

\bibitem{COS04}
M. Roughan, S. Sen, O. Spatscheck and N. Duffield, 
``Class-of-servicemapping for QoS: a statistical signature-based approach to IP traffic classification,''  In Proc. of ACM SIGCOMM Conference on Internet Measurement, Oct. 2004.

\bibitem{TFT14}
P. C. Lin, S. Y. Chen and C. H. Lin, 
``Towards Fine-grained Traffic Classification for Web Applications,'' In Proc. of Australasian Telecommunication Networks and Applications Conference (ATNAC), Nov. 2014.

\bibitem{TTC}
P. Schneider,
``TCP/IP traffic Classification Based on port numbers,'' Diploma Thesis, Division of Applied Science, Harvard University, 29 Oxford Street, Cambridge, MA 02138, USA, 1-6

\bibitem{TCI13}
Y. Xue, D. Wang and L. Zhang,
``Traffic Classification: Issues and Challenges,'' In Proc. of the International Conference on Computing, Networking and Communications (ICNC), Jan. 2013.

\bibitem{nDPI}
\url{http://www.ntop.org/products/deep-packet-inspection/ndpi/}

\bibitem{l7-filter}
\url{http://l7-filter.clearos.com/}

\bibitem{EMF14}
B. Hullar, S. Laki and A. Gyorgy,
``Efficient Methods for Early Protocol Identification,'' IEEE Journal on Selected Areas in Communications,  vol. 32, issue 10, Oct. 2014. 

\bibitem{EAI06}
L. Bernaille, R. Teixeira and K. Salamatian, 
``Early application identification,'' In Proc. of the Conference on Future Networking Technologies, 2006.

\bibitem{ATC13}
H. M. An, M. S. Kim and J. H. Ham,
``Application traffic classification using statistic signature,'' In Proc. of the 15th Asia-Pacific Network Operations and Management Symposium (APNOMS), Sept. 2013.

\bibitem{AIS14}
B. Schmidt, D. Kountanis and A. Al-Fuqaha, 
``Artificial Immune System Inspired Algorithm for Flow-Based Internet Traffic Classification,'' In Proc. of the IEEE 6th International Conference on Cloud Computing Technology and Science (CloudCom), Dec. 2014. 

\bibitem{FPN13}
S. Tapaswi and A. S. Gupta,
``Flow-Based P2P Network Traffic Classification Using Machine Learning,'' In Proc. of the International Conference on Cyber-Enabled Distributed Computing and Knowledge Discovery (CyberC), Oct. 2013. 

\bibitem{RTC10}
F .Dehghani, N .Movahhedinia, M. R. Khayyambashi and S. Kianian, 
``Real-Time Traffic Classification Based on Statistical and Payload Content Features,'' In Proc. of the 2nd International Workshop on Intelligent Systems and Applications (ISA), May 2010. 

\bibitem{ASA09}
S. Huang, K. Chen, C. Liu, A. Liang and H. Guan,
``A statistical-feature-based approach to internet traffic classification using Machine Learning,'' In Proc. of the International Conference on Ultra Modern Telecommunications and Workshops (ICUMT), Oct. 2009. 

\bibitem{FTC09}
M. Kohara, T. Hori, K. Sakurai, H. Lee and J. C. Ryou,
``Flow Traffic Classification with Support Vector Machine by Using Payload Length,'' In Proc. of the 2nd International Conference on Computer Science and its Applications (CSA), Dec. 2009.  

\bibitem{CMF04}
M. Roughan, S. Sen, O. Spatscheck and N. Duffield, 
``Class-of-Service Mapping for Qos: A Statistical Signature-based Approach to IP Traffic Classification,'' In Proc. of the Internet Measurement Conference (IMC), Oct. 2004.

\bibitem{COC12}
J. Zhang, C. Chen, Y. Xiang and W. Zhou, 
``Classification of Correlated Internet Traffic Flows,'' In Proc. of the 11th IEEE International Conference on Trust, Security and Privacy in Computing and Communications (TrustCom), June 2012.

\bibitem{FBC11}
S. Valenti and D. Rossi, 
``Fine-grained behavioral classification in the core: the issue of flow sampling,'' In Proc. of the 7th International Wireless Communications and Mobile Computing Conference (IWCMC), July 2011.
 
\bibitem{TFT11}
B. Park, J. W. Hong and Y. J. Won,
``Toward Fine-Grained Traffic Classification,'' IEEE Communication Magazine, vol. 49, issue. 7, pp. 104-111, July 2011. 
 
\bibitem{OTP11}
B. Augustin and A. Mellouk,
``On Traffic Patterns of HTTP Applications,'' In Proc. of the IEEE Global Communications Conference (GLOBECOM), Dec. 2011. 
 
\bibitem{DHC11}
R. Archibald, Y. Liu, C. Corbett and D. Ghosal, 
``Disambiguating HTTP: Classifying web Applications,'' In Proc. of the 7th International Wireless Communications and Mobile Computing Conference (IWCMC), July 2011.

\bibitem{MAM13}
P. Casas and P. Fiadino, 
``Mini-IPC: A minimalist approach for HTTP traffic classification using IP addresses,'' In Proc. of the 9th International Wireless Communications and Mobile Computing Conference (IWCMC), July 2013.
 
\bibitem{libnids}
\url{http://libnids.sourceforge.net}.  

\bibitem{AAM10}
M. Lee, R. R. Kompella and S. Singh, 
``Ajaxtracker: active measurement system for high-fidelity characterization of Ajax applications,'' In Proc. of the 2010 USENIX conference on Web application development (WebApps '10), 2010.  

\bibitem{MTC09}
S. Veres and D. Ionescu, 
``Measurement-based traffic characterization for Web 2.0 applications,'' In Proc. of the IEEE Instrumentation and Measurement Technology Conference (I2MTC), May 2009.
 
\bibitem{COH08}
L. Shuai, G. Xie and J. Yang,
``Characterization of HTTP behavior on access networks in Web 2.0,'' In Proc. of the International Conference on Telecommunications (ICT), June 2008.

\bibitem{WTM09}
S. Lin, Z. Gao and K. Xu, 
``Web 2.0 traffic measurement: analysis on online map applications,'' In Proc. of the 18th International Workshop on Network and Operating Systems Support for digital audio and video, ser. NOSSDAV ’09, 2009.

\bibitem{HTTP2}
M. Belshe, BitGo, R. Peon and M. Thomson, Ed., ``Hypertext Transfer Protocol Version 2 (HTTP/2),'' RFC 7540, May 2015.

\bibitem{IBR}
\url{http://internet-browser-review.toptenreviews.com/}.

\bibitem{BS}
\url{http://www.w3schools.com/browsers/browsers_stats.asp}.

\bibitem{TNW08}
F. Schneider, S. Agarwal, T. Alpcan and A. Feldmann, `` The New Web: Characterizing AJAX Traffic,'' In Proc. of the Passive and Active Measurement Conference, 2008.

\bibitem{DM}
``The Data Mining and Knowledge Discovery Handbook,''
In Oded Maimon and Lior Rokach Tel-Aviv (eds.), University Israel, pp. 40-41.

\end{thebibliography}
\end{document}