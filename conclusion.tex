\chapter{Conclusion and future work}
\label{conclusion}
In this work, we discuss the hazard that a compromised switch in SDN may bring. We propose an effective way to scan through the entire network for disobedient forwarding behavior and evaluate it with various network topology types, scales and number of entries on each switch. The experimental result demonstrates that our method is effective in a network topology with balanced structure. The method detects through all the entries in topology ``fat\_tree\_4'' with around 135 packets, topology ``two\_tier\_10\_10'' with around 150 packets and topology ``three\_tier\_5\_5\_2'' with around 180 packets. Also, among the chosen common network topologies, the scale of the network topology does not affect the aggregation rates significantly, and the effective aggregation rate grows to 3.48 in a fat-tree topology once there are 120 entries on each switch. 

In future work, more realistic factors should be taken into consideration, such as the collaboration of multiple compromised switches and the entries with multiple match fields and wildcard fields. Furthermore, the aggregated group finding method performs DFS without any estimation, and the core algorithm can be improved by adding heuristics. Also, as new versions of OpenFlow are widely implemented, the method should be able to be optimized with the support of new features. For example, the auxiliary entries can be installed on the egress table once it is fully supported.