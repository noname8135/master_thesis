\chapter{Conclusion and future work}
\label{conclusion}
In this work, we discuss the hazard that an compromised switch in SDN network may bring. In the scenario we set up, we can see how the problems can be solved by using SDN features. Even the scenario is different, we believe that the new problem it introduced can also be solved with some extension and adjustment with the functionality SDN provides. We propose an effective way that is able to scan through the entire network for disobedient forwarding behavior with less number of packets, and evaluate it under different network topology type, scale and entry number on each switch.

The experimental result demonstrates that our method is effective under a network topology with balance structure. Also, the scale of the network topology does not effect the effeciency significantly, and the aggregation rate grows to 3.48 in fat tree topology once there are 120 entries in each switch. 

In the future work, more realistic factors should be take into consideration, such as the collaboration of multiple compromised switches, the entries with multiple match fields and wildcard fields. Furthermore, the aggregated group finding method currently performs DFS without any estimation, the core algorithm can be improve by adding heuristic function. Also, as the new versions of OpenFlow release, the method should able to be optimized with the support of the new features. For example, the auxiliary entries can be installed on egress table once it is fully supported.