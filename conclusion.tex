\chapter{Conclusion and future work}
\label{conclusion}
In this work, we can see how the problems in a certain scenario can be solved by using SDN features. Even the scenario is different, we believe that the new problem it introduced can also be solved with some extension and adjustment, as long as we are using SDN.


We can only try to anticipate what the attackers may try to target with SDNs.  The deployments are new, the protocols are new, the controller software is new, and the history of past SDN attacks is unknown.  Based on the SDN architecture, we can predict where an attacker may be likely to strike.  If we put ourselves in the attacker’s shoes, we might be able to spot a weakness to exploit. Then we can harden that weakness ahead of time.

Before an organization embarks on an SDN deployment project, they should consider how they will secure the system during the early design stage. 

In the future work, 

1. we may also want to figure out a way to detect other types of flow entries.

2. We might also want to take collaboration of multiple compromised switches into consideration.

3. Optimize the algorithm by improving the group finding method. Currently, the group finding method is not the bestXXXXXXX.

4. Considering the situation of multiple flow table and more complex match field condition