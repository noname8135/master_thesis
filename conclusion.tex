\chapter{Conclusion and future work}
\label{conclusion}
In this work, we discuss the hazard that a compromised switch in SDN may bring. We propose an effective way to scan through the entire network for disobedient forwarding behavior with fewer packets \red{state the number numerically}, and evaluate it with various network topology types, scales and number of entries on each switch. The experimental result demonstrates that our method is effective in a network topology with balanced structure. Also, the scale of the network topology does not affect the efficiency significantly, and the aggregation rate grows to 3.48 in a fat-tree topology once there are 120 entries in each switch. 

We can only anticipate what attackers may try to target at SDN. The deployments are new, the protocols are new, the controller software is new, and the history of past SDN attacks is unknown. Based on the SDN architecture, we can predict where an attacker may be likely to strike. If we put ourselves in the attacker\textquotesingle s shoes, we may be able to spot a weakness to exploit. Then we can harden that weakness ahead of time. Before an organization embarks on SDN deployment, they should consider how they would secure the system during the early design stage. \red{What's the key point of this paragraph? The conclusion is too broad, and has little to do with your work.}

In the future work, more realistic factors should be taken into consideration, such as the collaboration of multiple compromised switches and the entries with multiple match fields and wildcard fields. Furthermore, the aggregated group finding method performs DFS without any estimation, and the core algorithm can be improved by adding heuristics. Also, as the new versions of OpenFlow are released, the method should be able to be optimized with the support of new features. For example, the auxiliary entries can be installed on the egress table once it is fully supported.