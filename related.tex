\chapter{Background and related work}

\section{Software-defined network}
\label{sec:payload}


-------------------------
Software defined network (SDN) is a dynamic, manageable, cost-effective, and adaptable next-generation network structure. It is designed for dealing with large scale, complex, dynamic data center network existing today. It has the following characteristics: (1) Software programmable: One can configure, manage, and optimize network resources very quickly via dynamic, automated SDN programs. (2) Agile: Abstracting control from forwarding lets administrators dynamically adjust network-wide traffic flow to meet changing needs. (3) Centralized Policy Control: Network intelligence is centralized in software-based SDN controllers that maintain a global view of the network (4) Manageability: Due to network programmability, companies can develop new applications that can further improve manageability, integration, communication and so on. (5) Open standards-based and vendor-neutral: SDN simplifies network design and operation because instructions are provided by SDN controllers instead of multiple, vendor-specific devices and protocols\cite{SDN_WIKI}. To sum up, SDN provides greater visibility into network, reduces manual intervention, improves maintainability, and requires less hardware budget.

SDN Application (SDN App)
SDN Applications are programs that explicitly, directly, and programmatically communicate their network requirements and desired network behavior to the SDN Controller via a northbound interface (NBI). In addition they may consume an abstracted view of the network for their internal decision making purposes. An SDN Application consists of one SDN Application Logic and one or more NBI Drivers. SDN Applications may themselves expose another layer of abstracted network control, thus offering one or more higher-level NBIs through respective NBI agents.
SDN Controller
The SDN Controller is a logically centralized entity in charge of (i) translating the requirements from the SDN Application layer down to the SDN Datapaths and (ii) providing the SDN Applications with an abstract view of the network (which may include statistics and events). An SDN Controller consists of one or more NBI Agents, the SDN Control Logic, and the Control to Data-Plane Interface (CDPI) driver. Definition as a logically centralized entity neither prescribes nor precludes implementation details such as the federation of multiple controllers, the hierarchical connection of controllers, communication interfaces between controllers, nor virtualization or slicing of network resources.
SDN Datapath
The SDN Datapath is a logical network device that exposes visibility and uncontended control over its advertised forwarding and data processing capabilities. The logical representation may encompass all or a subset of the physical substrate resources. An SDN Datapath comprises a CDPI agent and a set of one or more traffic forwarding engines and zero or more traffic processing functions. These engines and functions may include simple forwarding between the datapath's external interfaces or internal traffic processing or termination functions. One or more SDN Datapaths may be contained in a single (physical) network element—an integrated physical combination of communications resources, managed as a unit. An SDN Datapath may also be defined across multiple physical network elements. This logical definition neither prescribes nor precludes implementation details such as the logical to physical mapping, management of shared physical resources, virtualization or slicing of the SDN Datapath, interoperability with non-SDN networking, nor the data processing functionality, which can include L4-7 functions.
SDN Control to Data-Plane Interface (CDPI)
The SDN CDPI is the interface defined between an SDN Controller and an SDN Datapath, which provides at least (i) programmatic control of all forwarding operations, (ii) capabilities advertisement, (iii) statistics reporting, and (iv) event notification. One value of SDN lies in the expectation that the CDPI is implemented in an open, vendor-neutral and interoperable way.
SDN Northbound Interfaces (NBI)
SDN NBIs are interfaces between SDN Applications and SDN Controllers and typically provide abstract network views and enable direct expression of network behavior and requirements. This may occur at any level of abstraction (latitude) and across different sets of functionality (longitude). One value of SDN lies in the expectation that these interfaces are implemented in an open, vendor-neutral and interoperable way.


\section{OpenFlow structure}
\label{sec:flow}
It enables network controllers to determine the path of packets by managing forwarding tables in the switches\cite{OF_WIKI}. In this paper, we will be using Open vSwitch, a software implementation of virtual multilayer network switch that supports OpenFlow protocol. 

Another expanding set of classification techniques are based on machine learning with statistical features of network traffic. The studies in \cite{EAI06, ATC13} classify the behavior of an application by the size and the direction of the first few packets of a TCP connection. Bernaille et al. \cite{EAI06} defines the behavior of an application as the size and direction of the first four packets it exchanges over a TCP connection. An et al. \cite{ATC13} uses the first five packets and converts the flow into a vector. The paper shows the possibility of classifying traffic using the payload size distribution of the packets; nevertheless, this method classifies all application traffic, except HTTP.

Some studies \cite{AIS14, CMF04, COC12} extract the features from a complete flow. Common features are port numbers, average segment size, internal time of packets, and so on. \cite{AIS14} uses a Fast Correlation-Based Filter (FCBF) to choose the essential feature, which can reduce the processing time. 

Furthermore, some studies \cite{FPN13, RTC10, AIS14, ASA09, FTC09} used clustering techniques in machine learning to classify traffic. \cite{FPN13, RTC10} are based on Naive Bayes algorithm, a supervised machine learning algorithm based on Baysian theory. Artificial Immune System (AIS) algorithm is proven to be very versatile and low sensitivity to input parameters, so \cite{AIS14} used it as their base. The study implemented the features as hyper spheres within an 11-dimensional space, and if an example vector falls within the distance of the hyper sphere, then the example is classified to the same class as the feature belong to. The rest of studies are based on k-Nearest Neighbor (KNN) estimator \cite{ASA09} and on support vector machine (SVM) \cite{FTC09}. Machine-learning-based classification can automatically build a model from training data and reduce much artificial input; however, the classification will decrease because they only collect and classify traffic in a limited range. The limitation is a challenge. It is difficult to identify all protocols by using only a machine learning model because most models are binary classification, which means one model just can classify one protocol.


As enterprises look to adopt Software Defined Networking (SDN), the top of mind issue is the concern for security. Enterprises want to know how SDN products will assure them that their applications, data and infrastructure will not be vulnerable. With the introduction of SDN, new strategies for securing the control plane traffic are needed. This article will review the attack vectors of SDN systems and share ways to secure the SDN-enabled virtualized network infrastructure. This article will then discuss the methods currently being considered to secure SDN deployments.

1. SDN Attack Vectors

Software-Defined Networking (SDN) is an approach to networking that separates the control plane from the forwarding plane to support virtualization.  SDN is a new paradigm for network virtualization.  Most SDN architecture models have three layers: a lower layer of SDN-capable network devices, a middle layer of SDN controller(s), and a higher layer that includes the applications and services that request or configure the SDN.  Even though many SDN systems are relatively new and SDN is still in the realm of the early adopters, we can be sure, that as the technology matures and is more widely deployed, it will become a target for attackers.

We can anticipate several attack vectors on SDN systems.  The more common SDN security concerns include attacks at the various SDN architecture layers.  Let’s look at the anticipated attacks that could occur at each of these layers.  Following is a picture to illustrate a typical SDN architecture and where attackers may be coming from.
1-1. Attacks at Data Plane Layer

Attackers could target the network elements from within the network itself.  An attacker could theoretically gain unauthorized physical or virtual access to the network or compromise a host that is already connected to the SDN and then try to perform attacks to destabilize the network elements.  This could be a type of Denial of Service (DoS) attack or it could be a type of fuzzing attack to try to attack the network elements.

There are numerous southbound APIs and protocols used for the controller to communicate with the network elements.  These SDN southbound communications could use OpenFlow (OF), Open vSwitch Database Management Protocol (OVSDB), Path Computation Element Communication Protocol (PCEP), Interface to the Routing System (I2RS), BGP-LS, OpenStack Neutron, Open Management Infrastructure (OMI), Puppet, Chef, Diameter, Radius, NETCONF, Extensible Messaging and Presence Protocol (XMPP), Locator/ID Separation Protocol (LISP), Simple Network Management Protocol (SNMP), CLI, Embedded Event Manager (EEM), Cisco onePK, Application Centric Infrastructure (ACI), Opflex, among others.  Each of these protocols has their own methods of securing the communications to network elements.  However, many of these protocols are very new and implementers may not have set them up in the most secure way possible.

An attacker could also leverage these protocols and attempt to instantiate new flows into the device’s flow-table.  The attacker would want to try to spoof new flows to permit specific types of traffic that should be disallowed across the network.  If an attacker could create a flow that bypasses the traffic steering that guides traffic through a firewall the attacker would have a decided advantage.  If the attacker can steer traffic in their direction, they may try to leverage that capability to sniff traffic and perform a Man in the Middle (MITM) attack.
1-2. Attacks at Controller Layer

It is obvious that the SDN controller is an attack target.  An attacker would try to target the SDN controller for several purposes.  The attacker would want to instantiate new flows by either spoofing northbound API messages or spoofing southbound messages toward the network devices.  If an attacker can successfully spoof flows from the legitimate controller then the attacker would have the ability to allow traffic to flow across the SDN at their will and possibly bypass policies that may be relied on for security.

An attacker might try to perform a DoS of the controller or use another method to cause the controller to fail. The attacker might try to attempt some form of resource consumption attack on the controller to bog it down and cause it to respond extremely slowly to Packet\_In events and make it slow to send Packet\_Out messages.

\section{SDN security}
\subsection{Controller}
\subsection{Host}
\subsection{Switch}
\subsection{Hardening strategies}
Lastly, it would be bad if an attacker created their own controller and got network elements to believe flows from the rogue controller.  The attacker could then create entries in the flow tables of the network elements and the SDN engineers would not have visibility to those flows from the perspective of the production controller.  In this case, the attacker would have complete control of the network.

1-3. Attacks at SDN Layer

Attacking the security of the northbound protocol would also be a likely vector.  There are many northbound APIs that are used by SDN controllers.  Northbound APIs could use Python, Java, C, REST, XML, JSON, among others.  If the attacker could leverage the vulnerable northbound API, then the attacker would have control over the SDN network through the controller.  If the controller lacked any form of security for the northbound API, then the attacker might be able to create their own SDN policies and thus gain control of the SDN environment.

Often times, there is a default password that is used for a REST API which is trivial to determine.  If an SDN deployment didn’t change this default password and the attacker could create packets toward the controller’s management interface, then the attacker could query the configuration of the SDN environment and put in their own configuration.
2. Hardening an SDN System

With the introduction of SDN, a new method is needed for securing the control plane traffic.  In traditional IP networks the control plane security came in the form of routing protocol security measures that involved using MD5 for EIGRP, IS-IS, or OSPFv2, IPsec AH in the case of OSPFv3, or GTSM/ACLs/passwords for MP-BGP.  Some implementers do not even follow these simple techniques for traditional IP networks.  If they approach deployment of an SDN with the same disregard for security, then they will be exposing their organization to attacks.  Let’s look at how one can secure an SDN system based on hardening the three layers illustrated in the above architecture diagram.

2-1. Securing the Data Plane Layer

Typical SDN systems leverage X86 processors and use TLS (formerly SSL) for the security of the control plane.  These long-lived HTTP sessions are susceptible to a wide range of attacks that could jeopardize the integrity of the data plane.  This would bypass the multi-tenancy of these solutions and cause cloud-based services to be compromised.  Organizations should prefer to use TLS to authenticate and encrypt traffic between network device agent and controller.  Using TLS helps to authenticate controller and network devices/SDN agent and avoid eavesdropping and spoofed southbound communications.

Depending on the southbound protocol being used, there may be options to secure this communications.  Some protocols may be used within TLS sessions as previously mentioned.  Other protocols may use shared-secret passwords and/or nonce to prevent replay attacks.  Protocols like SNMPv3 offer more security than SNMPv2c and SSH is far better than Telnet.  Other proprietary southbound protocols may have their own methods to authenticate network device agents and controllers and encrypt data between themselves, thus thwarting the attacker’s eavesdropping and spoofing.

Similarly, depending on the Data Center Interconnect (DCI) protocol being used, there may be configurable options to authenticate tunnel endpoints and secure tunneled traffic.  Again, passwords/shared-secrets might be an option.  However, some DCI protocols may not have any option for security.

Organizations may believe that a private network has certain inherent security.  As organizations extend their virtual networks and SDNs to cloud services and to remote data centers, verifying the physical path may not be so easy.  Preventing unauthorized access is easier when an organization controls the physical access, but as networks virtualize, the actual physical path gets a little murky.  It is difficult to secure what you can’t see.

2-2. Securing the Controller Layer

The controller is a key attack target and therefore, it must be hardened.  Hardening the security posture of the controller and the network elements typical comes down to host OS hardening.  All the best practices for hardening public-facing Linux servers are applicable here.  Still, organizations will will want to closely monitor their controllers for any suspicious activity.

Organizations will also want to prevent unauthorized access to SDN control network.  SDN systems should allow for configuration of secure and authenticated administrator access to controller.  Even Role-Based Access Control (RBAC) policies may be required for controller administrators.  Logging and audit trails could be useful for checking for unauthorized changes by administrators or attackers alike.

If there is a DoS attack of the controller, then it is beneficial to have a High-Availability (HA) controller architecture.  SDNs that use redundant controllers could suffer the loss of a controller and continue to function.  This would raise the bar for an attacker trying to DoS all the controllers in the system.  Besides, that attack would not be particularly stealthy and further the attacker’s aims of remaining undetected.

2-3. Securing the SDN Layer

Another protection measure is to use an Out-of-Band (OOB) network for control traffic.  It is easier and less costly to construct an OOB network in a data center than across an enterprise WAN.  Using an OOB network for the northbound and southbound communications could help secure the protocols for controller management.

Using TLS or SSH or another method to secure northbound communications and secure controller management would be considered a best practice.  The communications from the applications and services requesting services or data from the controller should be secured using authentication and encryption methods.

Secure coding practices for all northbound applications requesting SDN resources should be a best practice.  Not only are secure coding practices beneficial to the security of public-facing Internet web applications, but they are also applicable to northbound SDN connections.

There are some SDN systems that have the ability to validate flows in network device tables against controller policy.  This type of checking (similar to FlowChecker) of the flows in the network devices against the policy in the policy could help identify discrepancies that are the result of an attack.

3. Summary

We can only try to anticipate what the attackers may try to target with SDNs.  The deployments are new, the protocols are new, the controller software is new, and the history of past SDN attacks is unknown.  Based on the SDN architecture, we can predict where an attacker may be likely to strike.  If we put ourselves in the attacker’s shoes, we might be able to spot a weakness to exploit.  Then we can harden that weakness ahead of time.

Before an organization embarks on an SDN deployment project, they should consider how they will secure the system during the early design stage.  Don’t leave security until the final clean-up phase.  If an organization waits until it is working, then hardening the northbound and southbound control messages may cause service-affecting problems.  Like most things, setting it up right from the start will save organizations many problems down the road.\cite{SSAV_NWW}






\section{}
\label{sec:grained}
The main idea behind fine-grained traffic classification is how to categorize application traffic into different traffic groups, and it has some advantages to improve the network security and the QoS management. There are some instances below:

S. Valenti et al. presented \cite{FBC11}, which use the Abacus behavioral classification engine and SVM to categorize the P2P traffic. They apply the traces of the six P2P applications, ranging from file-sharing to VoIP and live-streaming services, such as SopCast, Skype and BitTorrent. The research use Abacus classifier by only relying on the count of packets and bytes peers exchange during fixed-length time-windows, but the accuracy would be decreased when the vantage points moves from the edge.

In \cite{TFT11}, the authors select P2P applications for verification since the various behavior can strongly represents the complexity of Internet applications. P2P applications provide many functions: web browsing, searching, downloading, messenger and commercial advertisement. This paper uses Fileguri and BitTorrent to generate the flows, and creates the signatures by fine-grained classifier and LASER. They collected 450Gbytes from their campus network, then transform each packet into a vector, and exploited Jaccard for clustering.


------------------------------------
\
\begin{table}[H]
\centering
\caption{Pros and cons of each classification.}
\begin{tabular}{|l|p{3.5cm}|p{3.5cm}|p{3.5cm}|}
\hline Classification & Features & Pros and Cons & Used in studies \\
\hline
\hline Port-based & well-known port numbers & simply and fast but can't be used in the tunnelling traffic & \cite{TTC}\\
\hline Payload-based &  string-based matching, regular expression matching and IP protocol & accurate classification capability and easy development but slow and can't be used in encrypted traffic & \cite{nDPI} , {l7-filter} \\
\hline flow-based & packet size, packet inter-arrival time, flow duration, packet numbers,etc. & can detect the class of yet unknown applications but cannot deal with multiple transport transport protocols simultaneously & \cite{EAI06}, \cite{ATC13}, \cite{AIS14}, \cite{CMF04}, \cite{COC12}, \cite{FPN13, RTC10, AIS14, ASA09, FTC09} \\
\hline 
\end{tabular}
\label{table:pros-cons}
\end{table}
