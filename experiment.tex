\chapter{Detection Method Implementaion and Evaluation}
In this chapter, we detail experimental detail of methods we proposed in chapter 3. First, the tools and config settings will be listed. Then, we will 

\section{General experimental setup}
We use Mininet as our network environment. It allows us to creates a large-scale virtual network easily. 
It provides python API as well as command line interface to customize the network. It also offers an interactive interface to test the connectivity and performance. We use the virtual machine provided by the OpenFlow tutorial \cite{OFT}. It comes with all the elements we need, such as Mininet, ryu controller, OpenVswitch etc.

Table~\ref{Experiment table} is a summary of all the tools used. 

In both of our methods, hosts are irrevelant, so there will not be any host in our environment. Some setup like topology or number of switches are different in two parts of our work, they will be described individually.

\begin{table}[H]
\centering
\caption{Experimental environment summary}
\begin{tabular}{|l|p{4cm}|p{4.5cm}}
\hline Item & Detail version \\
\hline Operating system & Ubuntu 12.10 x86\_64 \\
\hline Controller & ryu\_manager 3.6 \\
\hline Network Emulator & Mininet 2.1.0\\
\hline Southbound API & OpenFlow 1.0.0 \\
\hline Virtual switch & OpenvSwitch 1.11.0 \\
\hline 
\end{tabular}
\label{table:Experiment table}
\end{table}


\section{Method Implementation}
preinstall flow entries proactively to simulate the . 
\subsection{}



\section{Evaluation}


there will be only \textit{output to port} actions among all the flow entries of all the switches in our environment. \ref{flow_entry_verification_attack_scenario} 