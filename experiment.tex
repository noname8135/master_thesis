\chapter{Detection Method Implementation and Evaluation}
In this chapter, we detail experimental detail of methods we proposed in chapter 3. First, the environment and the reason of using them will be explained. Then, we will show test items and their correspondent result.

\section{General experimental setup}
We use Mininet as our network environment. It allows us to creates a large-scale virtual network easily. 
It provides python API as well as command line interface to customize the network. It also offers an interactive interface to test the connectivity and performance. We use the virtual machine provided by the OpenFlow tutorial \cite{OFT}. It comes with all the elements we need, such as Mininet, ryu controller, OpenvSwitch etc.

Table~~\ref{table:Experiment_table} is a summary of all the tools used. 

In both of our methods, hosts are irrelevant, so there will not be any host in our environment. Some setup like topology or number of switches are different in two parts of our work, they will be described individually.

\begin{table}[H]
\centering
\caption{Experimental environment summary}
\begin{tabular}{|l|p{4cm}|p{4.5cm}}
\hline Item & Detail version \\
\hline Operating system & Ubuntu 14.04 x86\_64 \\
\hline Controller & ryu\_manager 4.0 \\
\hline Network Emulator & Mininet 2.2.1\\
\hline Southbound API & OpenFlow 1.3 \\
\hline Virtual switch & OpenvSwitch 2.5.0 \\
\hline 
\end{tabular}
\label{table:Experiment_table}
\end{table}

\section{Method Implementation}
preinstall flow entries proactively to simulate the situation. 
In the experiment, there will be only \textit{output to port} actions among all the flow entries of all the switches in our environment. 
\subsection{Flow entry verification method}
There will be only output actions among all the flow entries of all the switches in our experimental environment.


There are 253 flow tables in every switch simulated by mininet, we will only use the first table.

Since compromised switch is able to send fake xxxx to controller, controller cannot rely on the response of the switches. Hence, it needs to maintain detail of flow entry it installed. 

\section{Evaluation}


there will be only \textit{output to port} actions among all the flow entries of all the switches in our environment. \ref{flow_entry_verification_attack_scenario} 